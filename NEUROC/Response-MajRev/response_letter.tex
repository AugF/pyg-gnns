% !TEX program=pdflatex
% LaTeX rebuttal letter example. 
% 
% Copyright 2019 Friedemann Zenke, fzenke.net
%
% Based on examples by Dirk Eddelbuettel, Fran and others from 
% https://tex.stackexchange.com/questions/2317/latex-style-or-macro-for-detailed-response-to-referee-report
% 
% Licensed under cc by-sa 3.0 with attribution required.

\documentclass[12pt]{article}
\usepackage[utf8]{inputenc}
\usepackage{lipsum} % to generate some filler text
\usepackage{fullpage}
\usepackage{times}
\usepackage{amsmath, amsthm, amsfonts}
\usepackage{booktabs}
\usepackage{makecell}
\usepackage{multirow}
\usepackage{lscape}
\usepackage{xcolor}
\usepackage{float}
% Vector
\newcommand{\MyVec}[1]{\boldsymbol{#1}}
% Matrix
\newcommand{\MyMat}[1]{\boldsymbol{#1}}
% Vectors with hat
\newcommand{\hvec}[1]{\hat{\boldsymbol{h}}_{#1}}
% Model parameters
\newcommand{\Param}[1]{\textcolor{blue}{#1}}


% import Eq and Section references from the main manuscript where needed
% \usepackage{xr}
% \externaldocument{manuscript}

% package needed for optional arguments
\usepackage{xifthen}
% define counters for reviewers and their points
\newcounter{reviewer}
\setcounter{reviewer}{0}
\newcounter{point}[reviewer]
\setcounter{point}{0}

% This refines the format of how the reviewer/point reference will appear.
\renewcommand{\thepoint}{P\,\thereviewer.\arabic{point}} 

% command declarations for reviewer points and our responses
\newcommand{\reviewersection}{\stepcounter{reviewer} \bigskip \hrule
                  \section*{Reviewer \thereviewer}}

\newenvironment{point}
   {\refstepcounter{point} \bigskip \noindent {\textbf{Reviewer~Point~\thepoint} } ---\ \begin{sf}}
   {\end{sf} \par}

\newcommand{\shortpoint}[1]{\refstepcounter{point}  \bigskip \noindent 
	{\textbf{Reviewer~Point~\thepoint} } ---\begin{sf} ~#1 \end{sf}\par}

\newenvironment{reply}
   {\medskip \noindent \textbf{Reply}:\  }
   {\medskip}

\newcommand{\shortreply}[2][]{\medskip \noindent \textbf{Reply}:\  #2
	\ifthenelse{\equal{#1}{}}{}{ \hfill \footnotesize (#1)}%
	\medskip}

\begin{document}

\section*{Response to the Reviewers}
% General intro text goes here
We thank the reviewers for their critical assessment and insightful comments of our work.
%
We have made extensive modifications to our manuscript.
%
In the following we address their concerns point by point.
%
We also prepare an \emph{annotated version} for our revised manuscript.
%
In the annotated version, changes corresponding to each point are all highlighted by red squares.

\subsection*{Summary of Changes}

%%%% TODO %%%

% Let's start point-by-point with Reviewer 1
\reviewersection

% R1Q0
\begin{point}
    The authors conduct an empirical analysis of performance bottlenecks in graph neural network training. The authors identify the edge-related calculation is the performance bottleneck. Experimental of several GNN variants, such as GCN, GGNNNN, GAT and GaAN on six real-world graph datasets verify the importance of the findings. The experimental analysis is sufficient. However, there are some tiny issues in this paper.
\end{point}

\begin{reply}
    Thank you for your positive comments on our manuscript.
    %
    We have carefully revised the manuscript according to your comments. 
    %
    Based on the suggestions, we have made an extensive modification on the revised manuscript.
    %
    Please see the detailed responses below.
    %
    The changes corresponding to each issue were highlighted by red squares in the annotated version of our manuscript.

\end{reply}

% R1Q1
\begin{point}
	There are lots of symbols in this paper. Some symbols are reused and confusing, such as s denotes sub-layers or edge features.
\end{point}

% Our reply
\begin{reply}
    We apologize for the confusing use of symbols and thank you for pointing the problem out.
    %
    To clarify the symbol usage, we have checked the manuscript and unified the usage of symbol.
    %
    We avoid the problem of reusing symbols, so that each symbol only represents one meaning after the revision.
    %
    We summarize the frequently-used symbols in Table 1 in the revised manuscript.
    %
    We quote the table below:
    \begin{table}[H]
        \footnotesize
        \centering
    \begin{tabular}{p{3em}lp{35em}}
        \toprule
        Category & Symbol & Meaning \\
        \midrule
        \multirow[c]{4}{3em}{Graph Structure}& $\mathcal{G}=(\mathcal{V}, \mathcal{E})$ & The simple undirected input graph with the vertex set $\mathcal{V}$ and the edge set $\mathcal{E}$. \\
        & $v_x$ & The $x$-th vertex of the input graph. \\
        & $e_{x,y}$ & The edge pointing from $v_x$ to $v_y$ of the input graph. \\
        & $\mathcal{N}(v_x)$ & The adjacency set of $v_x$ in the input graph. \\ 
        & $\bar{d}$ & The average degree of the input graph. \\ \midrule
        \multirow[c]{6}{3em}{GNN Definition}& $L$ & The number of GNN layers. \\
        & $K$ & The number of heads in a GNN layer. \\
        & $\phi^l$ & The messaging function of the GNN layer $l$. \\
        & $\Sigma^l$ & The aggregation function of the GNN layer $l$. \\
        & $\gamma^l$ & The vertex updating function of the GNN layer $l$. \\ 
        & $\phi^{l,i}$ / $\Sigma^{l,i}$ / $\gamma^{l,i}$ & The messaging/aggregation/updating function of the $i$-th sub-layer of the GNN layer  $l$.\\
        & $\textcolor{blue}{\boldsymbol{W}^l, \boldsymbol{W}^{(k)}/\boldsymbol{b}, \boldsymbol{a}}$ & The matrices/vectors represented by the blue characters are the weight matrices/vectors that need to be learned in the GNN. \\  \midrule
        \multirow[c]{8}{3em}{Vector}& $\boldsymbol{v}_x$ & The feature vector of the vertex $v_x$. \\
        & $\boldsymbol{e}_{x,y}$ & The feature vector of the edge $e_{x,y}$.  \\
        & $\boldsymbol{h}_x^{l}$ &  The {input} hidden vector of the graph neuron corresponding to $v_x$ in the GNN layer $l$. \\
        & $\boldsymbol{h}_x^{l+1}$ &  The {output} hidden vector of the graph neuron corresponding to $v_x$ in the GNN layer $l$.\\
        & $\boldsymbol{m}_{x,y}^l$ & The message vector of the edge $e_{x,y}$ outputted by $\phi^l$ of the GNN layer $l$. \\
        & $\boldsymbol{s}_{x}^l$ & The aggregated vector of the vertex $v_x$ outputted by $\Sigma^l$ of the GNN layer $l$. \\
        & $\boldsymbol{h}_{x}^{l,i}$ / $\boldsymbol{m}_{x,y}^{l,i}$ / $\boldsymbol{s}_{x}^{l,i}$ & The hidden/message/aggregated vector of the vertex $v_x$ outputted by $\gamma^{l,i}$/$\phi^{l,i}$/$\Sigma^{l,i}$ of the $i$-th sub-layer of the GNN layer $l$. \\
        & $d^l_{in}$, $d^l_{out}$ &  The dimension of the input/output hidden vectors of the GNN layer $l$. \\
        & $dim(\MyVec{x})$ & The dimension of a vector $\MyVec{x}$. \\
        \bottomrule
    \end{tabular}
\end{table}

    
    In the revised manuscript, we use $e_{x,y}$ to represent an edge and use $\boldsymbol{e}_{x,y}$ to represent its input feature vector.
    %
    The input feature vectors of all edges are same for all GNN layers.
    %
    We use $\boldsymbol{s}$ to represent aggregated vectors outputted by the aggregation function $\Sigma$ in graph neurons.
    %
    For every vertex $v_x$, we use $\boldsymbol{s}^l_x$ to denote its aggregated vector in the GNN layer $l$.
    %
    If the GNN layer $l$ has sub-layers, $\boldsymbol{s}^{l,i}_x$ represents its aggregated vector in the $i$-th sub-layer.
    
    To clarify the concept of \emph{sub-layers} in a GNN layer, we have added more description on it in the revised manuscript.
    %
    We first introduce the concept of sub-layer in Section 2.2 ``Graph Neuron and Message-passing Model'' as:
    %
    \begin{quote}
        Some complex GNNs like GAT [6] and GaAN [7] use more than one message passing phase in each GNN layer.
        %
        We regard every message passing phase in a GNN layer as a \emph{sub-layer}.
        %
        We will give out more details on sub-layers when we introduce GAT.
    \end{quote}
    %
    We then use GAT as an example to elaborate on the concept of sub-layers in Section 2.3 ``Representative GNNs'' as:
    %
    \begin{quote}
        \newcommand{\GATCalcWeight}{\exp(LeakyReLU(\Param{\MyVec{a}}^T[\hvec{y}[k] \parallel \hvec{x}[k]]))}
        Each GAT layer consists of a vertex pre-processing phase and two sub-layers (i.e., message-passing phases).
        
        The vertex pre-processing phase calculates the attention vector $\hat{\boldsymbol{h}}^{l}_{x}$ for every vertex $v_x$ by $\hvec{x} = \parallel_{k=1}^K \Param{\MyMat{W}^l_{(k)}}\MyVec{h}^l_x$. We denote the attention sub-vector generated by the $k$-th head as $\hvec{x}[k]=\Param{\MyMat{W}^l_{(k)}}\MyVec{h}^l_x$.
        
        The first sub-layer of GAT (defined in Equation~\ref{eq:GAT-sub-layer-1}) uses the attention vectors to emit the attention weight vector $\boldsymbol{m}^{l,0}_{y,x}$ for every edge $e_{y,x}$ and aggregates the attention weight vectors for every vertex $v_x$ to get the weight sum vector $\MyVec{h}^{l,0}_x$.
        %
        \begin{equation}
            \footnotesize
            \label{eq:GAT-sub-layer-1}
            \begin{aligned}
                \MyVec{m}^{l,0}_{y,x} & = \phi^{l,0}(\MyVec{h}^l_y, \MyVec{h}^l_x, \MyVec{e}_{y,x}, \hvec{y}, \hvec{x}) = \parallel_{k=1}^{K}\GATCalcWeight, \\
                \MyVec{s}^{l,0}_{x} &= {\Sigma}_{v_y \in \mathcal{N}(v_x)}{\MyVec{m}^{l,0}_{y,x}}, \\
                \MyVec{h}^{l,0}_{x} &= \gamma^{l,0}(\MyVec{h}^l_x, \MyVec{s}^{l,0}_{x})  = \MyVec{s}^{l,0}_{x}.
            \end{aligned}
        \end{equation}
        %
        The second sub-layer of GAT (defined in Equation~\ref{eq:GAT-sub-layer-2}) uses the weight sum vectors to normalize the attention weights for every edge and aggregates the attention vectors $\boldsymbol{\hat{h}}^l_y$ with the normalized weights.
        %
        The aggregated attention vectors $\MyVec{s}^{l,1}_x$ are transformed by an activation function $\delta$ and are outputted as the hidden vectors of the current layer $\MyVec{h}^{l+1}_x$.
        %
        \begin{equation}
            \footnotesize
            \label{eq:GAT-sub-layer-2}
            \begin{aligned}
                \MyVec{m}^{l,1}_{y,x} &= \phi^{l,1}(\MyVec{h}^{l,0}_y, \MyVec{h}^{l,0}_x, \MyVec{e}_{y,x}, \hvec{y}, \hvec{x}) = \parallel_{k=1}^{K}\frac{\GATCalcWeight}{\MyVec{h}^{l,0}_x[k]}\hvec{y}[k], \\
                \MyVec{s}^{l,1}_x &= {\Sigma}_{v_y \in \mathcal{N}(v_x)} \MyVec{m}^{l,1}_{y,x}, \\
                \MyVec{h}^{l+1}_x = \MyVec{h}^{l,1}_x &= \gamma^{l,1}(\MyVec{h}^{l,0}, \MyVec{s}^{l,1}_x) = \delta(\MyVec{s}^{l,1}_x).
            \end{aligned}
        \end{equation}
    \end{quote}  
\end{reply}

% R1Q2
\begin{point}
    Some typical applications of GNNs should be included, such as video object segmentation [ref1], human-object interaction [ref2] and human-parsing [ref3].[1] Zero-shot video object segmentation via attentive graph neural networks,iccv 2019 [2] Learning human-object interactions by graph parsing neural networks, eccv 2018. [3] Hierarchical human parsing with typed part-relation reasoning, cvpr 2020.
\end{point}

\begin{reply}
	Thank you for pointing out our shortcomings. Computer vision is indeed another important application area of graph neural networks. We have added the mentioned references in the INTRODUCTION section in the revised manuscript. We quote the related sentence below:
    \begin{quote}
         The powerful expression ability makes GNNs achieve good accuracy in not only graph analytical tasks [8, 9, 10] (like node classification and link prediction) but also computer vision tasks (like human-object interaction [11], human parsing [12], and video object segmentation [13]).
    \end{quote}
\end{reply}

% R1Q3
\begin{point}
There are some grammar errors and typos:
\\
  - `Take the demo GNN in Figure 1(a) as  the example.'
\\
 - `to calculate the  output hidden vector $h^{l+1}$ of the current layer l, i.e., $h^{l+1}$ = $\gamma^l(h^l,s^l)$ The end-to-end training requires…'
\\
 - `Implementing it with the specially optimized basic operators on the GPU is a potential optimization'
\\
 - The sentences in the experimental section should be unified.
\end{point}

\begin{reply}

    Thank you for pointing them out.
    %
    We feel really sorry for our carelessness.
    %
    We have proofread our revised manuscript carefully to eliminate grammar errors and typos.
    %
    We have also unified the tenses of the sentences in Section 3 ``Evaluation Design'' and Section 4 ``Evaluation Results and Analysis''.
    %
    We use the past tense to describe experimental methods, results and what they indicate.
    %
    We only use the present tense in the sentences that Figure/Table X are the subjects of the sentences.

\end{reply}

% R1Q4
\begin{point}
    Figures 6 and 7 should be adjusted. The figures and fonts are too small.
\end{point}

\begin{reply}
    
    We appreciate your comment.
    %
    We have enlarged Figure 6 and Figure 7 in the revised manuscript.
    %
    Besides them, we have also enlarged fonts in other figures in the manuscript, to make sure that font sizes in figures are no less than the font size of figure captions.

\end{reply}

% R1Q5
\begin{point}
    In my view, computation efficiency is to describe the testing or validation process. Except for reporting and analyzing the training times, it is meaningful to discuss the inference time. This is also an important point of view for deep learning researchers to be concerned about.
\end{point}

\begin{reply}
    Thank you for the insightful comment and suggestion.
    %
    The efficiency of inference (including inference time and memory usage) is indeed important for deep learning researchers and engineers.
    %
    In the revised manuscript, we have added a new subsection (Section 2.5 ``Inference with GNNs'') in the revised manuscript to briefly review how to perform inference with GNNs.
    %
    To discuss the efficiency of GNN inference on GPUs, we have added performance bottleneck analysis on GNN inference in every subsection of Section 4 ``Evaluation Results and Analysis'':
    %
    \begin{enumerate}

        \item In Section 4.1 ``Effects of Hyper-parameters on Performance'', we have additionally measured how the inference time and the peak memory usage changed as we increased the values of the hyper-parameters.
        %
        We find that ...
        %%% TODO %%%
        
        \item In Section 4.2 ``Time Breakdown Analysis'', we have additionally conducted the time breakdown analysis for GNN inference.
        %
        We find that the performance characteristics of GNN inference are similar to GNN training.
        %
        The performance bottenecks in GNN inference are also similar to the bottlenecks in training.
        %
        Therefore, we focus on discussing the differences between GNN training and inference in the revised manuscript.
        
        \item In Section 4.3 ``Memory Usage Analysis'', we have additionally conducted the memory usage analysis for GNN inference.
        %
        We find that ... 
        %%% TODO %%%

        \item In Section 4.4 ``Effects of Sampling Techniques on Performance'', we have additionally evaluated the performance bottlenecks in sample-based GNN inference.
        %
        We find that ...
        %%% TODO %%%
    \end{enumerate}

\end{reply}

% Begin a new reviewer section
\reviewersection

\begin{point}
    In the paper, authors accomplished a unique study and analysis on GNN models training complexity.  The articles first review and development history of GNNs and creatively model all architectures as input layers, intermediate layers of graph neurons and prediction layers. And they quantitatively summarize the time and space complexity of 4 representative GNNs, including graph convolution, gated recurrent graph net, graph attention net and GraphSage. Most importantly, the article first break down complexity into operator level and offered analysis of good granularity, giving reader more guidance in future study. At last, the solid experiments included the study of effects of hyper-parameters and a comparison of two major sampling techniques: neighbor sampling and cluster sampling.
\end{point}

\begin{reply}
    Thank you for your positive comments on our manuscript.
    %
    We have carefully revised the manuscript according to your comments.
    %
    We have revised our manuscript according to your kindly suggestions.
    %
    Please see the detailed responses below.
    %
    We have highlighted our modification point by point in the annotated version of the manuscript by red squares.
\end{reply}

\begin{point}
    In general, the paper was well written and organized with good structure and clear narratives. Just some minor language errors like line Page 8, Line 208, "In active graph neurons" =\textgreater "Inactive graph neurons".
\end{point}

% R2Q1
\begin{reply}
    Thank you for pointing them out.
    %
    We feel really sorry for our carelessness.
    %
    We have proofread our revised manuscript carefully to eliminate such language errors as much as we can.
\end{reply}

% R2Q2
\begin{point}
    I was impressed by the way that authors categorize layers and operators in GNNs, very clear and instructive.
    
    It is also pretty neat to divide layer time complexity into two buckets: vertex calculation and edge calculation. The data model pretty well summarizes mainstream GNN layer architectures. And this analysis is very insightful for layer profiling.
    
    And the experimental evaluation were done over 6 large graph-structured datasets.
\end{point}

\begin{reply}
    Thank you very much for your appreciation.	
\end{reply}

% R2Q3
\begin{point}
    While, one major drawback is that I did not clearly see the analysis complexity v.s. accuracy. For example, in Figure 19 and 20, I did not see network accuracy from those 4 GNNs. There is always tradeoff between model complexity and model performance, and in some scenarios where high complexity is allowed, a sophisticated model of more powerful representation capability is still needed.
\end{point}

\begin{reply}
\end{reply}

\begin{point}
    Sampling method is definitely going to reduce model complexity, since all models complexity depend on graph node number N, while performance is going to be compromised as well. I would like to see authors resolve the concern of significant accuracy drop after applying aggressive sampling of subgraphs.
\end{point}

\begin{reply}
    
\end{reply}

\begin{point}
    Hope authors supplement the effect of sampling and GNNs on accuracy while comparing different complexity of model and sampling methods.
\end{point}

\begin{reply}
    
\end{reply}

\end{document}