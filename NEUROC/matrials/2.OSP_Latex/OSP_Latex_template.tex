%% 
%% Copyright 2007, 2008, 2009 Elsevier Ltd
%% 
%% This file is part of the 'Elsarticle Bundle'.
%% ---------------------------------------------
%% 
%% It may be distributed under the conditions of the LaTeX Project Public
%% License, either version 1.2 of this license or (at your option) any
%% later version.  The latest version of this license is in
%%    http://www.latex-project.org/lppl.txt
%% and version 1.2 or later is part of all distributions of LaTeX
%% version 1999/12/01 or later.
%% 
%% The list of all files belonging to the 'Elsarticle Bundle' is
%% given in the file `manifest.txt'.
%% 

%% Template article for Elsevier's document class `elsarticle'
%% with numbered style bibliographic references
%% SP 2008/03/01

\documentclass[preprint,12pt]{elsarticle}

%% Use the option review to obtain double line spacing
%% \documentclass[authoryear,preprint,review,12pt]{elsarticle}

%% For including figures, graphicx.sty has been loaded in
%% elsarticle.cls. If you prefer to use the old commands
%% please give \usepackage{epsfig}

%% The amssymb package provides various useful mathematical symbols
\usepackage{amssymb}
%% The amsthm package provides extended theorem environments
%% \usepackage{amsthm}

%% The lineno packages adds line numbers. Start line numbering with
%% \begin{linenumbers}, end it with \end{linenumbers}. Or switch it on
%% for the whole article with \linenumbers.
\usepackage{lineno}

\journal{Name of journal}

\begin{document}

\begin{frontmatter}

%% Title, authors and addresses

%% use the tnoteref command within \title for footnotes;
%% use the tnotetext command for theassociated footnote;
%% use the fnref command within \author or \address for footnotes;
%% use the fntext command for theassociated footnote;
%% use the corref command within \author for corresponding author footnotes;
%% use the cortext command for theassociated footnote;
%% use the ead command for the email address,
%% and the form \ead[url] for the home page:
%% \title{Title\tnoteref{label1}}
%% \tnotetext[label1]{}
%% \author{Name\corref{cor1}\fnref{label2}}
%% \ead{email address}
%% \ead[url]{home page}
%% \fntext[label2]{}
%% \cortext[cor1]{}
%% \address{Address\fnref{label3}}
%% \fntext[label3]{}

\title{Title/Name of your software}

%% use optional labels to link authors explicitly to addresses:
%% \author[label1,label2]{}
%% \address[label1]{}
%% \address[label2]{}

\author{A. Author}

\address{Your institute, some address}

\begin{abstract}
%% Text of abstract 
Ca. 100 words

\end{abstract}

\begin{keyword}
%% keywords here, in the form: keyword \sep keyword
keyword 1 \sep keyword 2 \sep keyword 3

%% PACS codes here, in the form: \PACS code \sep code

%% MSC codes here, in the form: \MSC code \sep code
%% or \MSC[2008] code \sep code (2000 is the default)

\end{keyword}

\end{frontmatter}

\linenumbers

%% main text

Description of your software in maximum 5 pages for first Original Software Publication –- see suggested format; 

\section{Introduction}
\label{}

Introduce the motivation of developing the software, and explain why it is important.

\section{Problems and Background}
\label{}

Give the formulations of problems to be solved by the software/toolbox.

Introduce the background and related work in literature (cite or list algorithms used, other software etc.).

\section{Software Framework }
\label{}

\subsection{Software Architecture}
\label{}

Give a short overview of the overall software architecture.

\subsection{Software Functionalities}
\label{}

Present the major functionalities of the software.

\subsection{Sample code snippets analysis (optional)}
\label{}

\section{Implementation and Empirical Results}
\label{}

Implementation details.

Empirical results.

Conduct empirical studies and provide results.

Compare with state-of-the-art software if any, kindly cite relevant work.

\section{Illustrative Examples}
\label{}

Provide at least one illustrative example to demonstrate the major functions.

Optional: you may include one explanatory video that will appear next to your article, in the right hand side panel. (Please upload any video as a single supplementary file with your article. Only one MP4 formatted, with 50MB maximum size, video is possible per article. Recommended video dimensions are 640 $\times$ 480 at a maximum of 30 frames/second. Prior to submission please test and validate your .mp4 file at $ http://elsevier-apps.sciverse.com/GadgetVideoPodcastPlayerWeb/verification$. This tool will display your video exactly in the same way as it will appear on ScienceDirect.).


\section{Conclusions}
\label{}

Set out the conclusion of this original software publication.

\section*{Acknowledgements}
\label{}

Optionally thank people and institutes you need to acknowledge. 

%% The Appendices part is started with the command \appendix;
%% appendix sections are then done as normal sections
%% \appendix

%% \section{}
%% \label{}

%% References: At least 5 are required 
%% If you have bibdatabase file and want bibtex to generate the
%% bibitems, please use
%%
%%  \bibliographystyle{elsarticle-num} 
%%  \bibliography{<your bibdatabase>}

%% else use the following coding to input the bibitems directly in the
%% TeX file.

\begin{thebibliography}{00}

%% \bibitem{label}
%% Text of bibliographic item

\bibitem{}

\end{thebibliography}

\clearpage
\section*{Required Metadata}
\label{}

\section*{Current executable software version}
\label{}

Ancillary data table required for sub version of the executable software: (x.1, x.2 etc.) kindly replace examples in right column with the correct information about your executables, and leave the left column as it is.

\begin{table}[!h]
\begin{tabular}{|l|p{6.5cm}|p{6.5cm}|}
\hline
\textbf{Nr.} & \textbf{(executable) Software metadata description} & \textbf{Please fill in this column} \\
\hline
S1 & Current software version & for example 1.1, 2.4 etc. \\
\hline
S2 & Permanent link to executables of this version  & example: $https://github.com/combogenomics/$ $DuctApe/releases/tag/DuctApe-0.16.4$ \\
\hline
S3 & Legal Software License & List one of the approved licenses \\
\hline
S4 & Computing platform/Operating System & for example Android, BSD, iOS, Linux, OS X, Microsoft Windows, Unix-like , IBM z/OS, distributed/web based etc. \\
\hline
S5 & Installation requirements \& dependencies & \\
\hline
S6 & If available, link to user manual - if formally published include a reference to the publication in the reference list & Example: $http://mozart.github.io/documentation/$ \\
\hline
S7 & Support email for questions & \\
\hline
\end{tabular}
\caption{Software metadata (optional)}
\label{} 
\end{table}

\section*{Current code version}
\label{}

Ancillary data table required for subversion of the codebase. Kindly replace examples in right column with the correct information about your current code, and leave the left column as it is.

\begin{table}[!h]
\begin{tabular}{|l|p{6.5cm}|p{6.5cm}|}
\hline
\textbf{Nr.} & \textbf{Code metadata description} & \textbf{Please fill in this column} \\
\hline
C1 & Current code version & For example v42 \\
\hline
C2 & Permanent link to code/repository used of this code version & For example: $https://github.com/mozart/mozart2$ \\
\hline
C3 & Legal Code License   & List one of the approved licenses \\
\hline
C4 & Code versioning system used & For example svn, git, mercurial, etc. put none if none \\
\hline
C5 & Software code languages, tools, and services used & For example c++, python, r,  etc. \\
\hline
C6 & Compilation requirements, operating environments \& dependencies & \\
\hline
C7 & If available Link to developer documentation/manual & For example: $http://mozart.github.io/documentation/$ \\
\hline
C8 & Support email for questions & \\
\hline
\end{tabular}
\caption{Code metadata (mandatory)}
\label{} 
\end{table}

\end{document}
\endinput
%%
%% End of file `OSP_Latex_template.tex'.
