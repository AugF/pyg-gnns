\section{Related Work}
\label{sec:related_work}

\paragraph{Survey of GNNs}
Zhou et al.\cite{zhou2018_gnn_review}, Zhang et al.\cite{zhang2018_gnn_survey} and Wu et al.\cite{comprehensive-survey-wu-2020} survey the existing graph neural networks and classify them from an algorithmic view.
They summarize the similarities and differences between the architectures of different GNNs.
The typical applications of GNNs are also briefly introduced.
Those surveys focus on comparing the existing GNNs theoretically, not empirically.

\paragraph{Evaluation of GNNs}
Shchur et al. \cite{shchur2018_pitfall_of_gnn} evaluate the accuracy of popular GNNs on the node classification task.
Dwived et al. \cite{dwivedi2020_benchmark_of_gnn} further compare the accuracy of popular GNNs fairly in a controlled environment.
Hu et al. \cite{hu2020_open_graph_benchmark} propose the open graph benchmark that provides a standard datasets and a standard evaluation workflow.
The benchmark makes comparisons between GNNs easily and fairly.
Those model evaluation efforts focus on evaluating the accuracy of different GNNs.
They provide insightful suggestions to improve accuracy.

From the efficiency aspect, Yan et al. \cite{yan2020_characterizing_gcn} compare the performance characteristics of graph convolutional networks, typical graph processing kernels(like PageRank), and the MLP-based neural networks on GPUs.
They provide optimization guidelines for both the software and the hardware developers.
Zhang et al. \cite{zhang2020_analysis_neugraph} analyze the architectural characteristics of the GNN inference on GPUs under SAGA-NN \cite{ma2019_neugraph} model.
They find that the GNN inference has no fixed performance bottleneck and all components deserve to optimize.
These two efforts focus on the \emph{inference} phase of GNNs and they investigate the potential optimizations mainly from an architectural view.
In this work, our target is to find out the performance bottleneck in the training phase from a system view.
We consider the performance bottleneck in both time and memory usage.
We also evaluate the effects of the sampling techniques.
Our work and the related evaluation\cite{yan2020_characterizing_gcn, zhang2020_analysis_neugraph} form a complementary study on the efficiency issue of GNNs.

\paragraph{Libraries/Systems for GNNs}

PyG \cite{PyG} and DGL \cite{DGL} both adopt the message-passing framework as the underlying programming model for GNNs and support training big datasets with the sampling techniques.
PyG \cite{PyG} is built upon PyTorch and it uses optimized CUDA kernels for GNNs to achieve high performance.
DGL \cite{DGL} provides a group of high-level user APIs and supports training GNNs with a variety of backends (TensorFlow, MXNet, and PyTorch) transparently.
It also supports LSTM as the aggregation functions.
NeuGraph \cite{ma2019_neugraph} proposes a new programming model \emph{SAGA-NN} for GNNs.
It focuses on training big datasets efficiently without sampling.
It partitions the dataset sophisticatedly, schedules the training tasks among multiple GPUs, and swaps the data among GPUs and the host asynchronously.
AliGraph \cite{zhu2019_aligraph} targets at training GNNs on big attributed heterogeneous graphs that are common in e-commerce platforms.
The graphs are partitioned among multiple nodes in a cluster and AliGraph trains GNNs on the graphs in a distributed way with system optimizations.
PGL \cite{PGL} is another graph learning framework from Baidu based on the PaddlePaddle platform.
