%! TEX root = ms.tex
%! TEX program = pdflatex
\documentclass{elsarticle}
\usepackage{geometry}
\geometry{left=3cm, right=3cm, top=2.5cm, bottom=2.5cm} 
\usepackage{lineno,hyperref}
\usepackage{amsmath, amsthm}
\usepackage{relsize}
\usepackage{booktabs}
\usepackage{makecell}
\usepackage{lscape}
\usepackage{xcolor}
\usepackage{subfig}
\usepackage{float}
\usepackage{times}
\modulolinenumbers[5]
\bibliographystyle{elsarticle-num}


\begin{document}

\begin{frontmatter}

	\title{Empirical Analysis of Performance Bottlenecks in Graph Neural Network Training with GPUs}
    \author{Zhaokang Wang, Yunpan Wang, Chunfeng Yuan, Rong Gu$^*$ \corref{correspondingauthor1}, Yihua Huang$^*$ \corref{correspondingauthor2}}
    \cortext[correspondingauthor]{Corresponding authors with equal contribution}
    \ead{\{wangzhaokang, wangyp\}@smail.nju.edu.cn, \{cfyuan, gurong, yhuang\}@nju.edu.cn}
	\address{State Key Laboratory for Novel Software Technology, \\Department of Computer Science and Technology, Nanjing University, \\Nanjing 210023, China}

	\begin{abstract}
		The graph neural network (GNN) has become a popular research area for its state-of-the-art performance in many graph analysis tasks. 
		Recently, various graph neural network libraries have emerged.They make the development of GNNs convenient, but their performance on large datasets is not satisfying. 
		In this work, we analyze the performance bottleneck in training GNN with GPUs empirically. 
		A GNN layer can be decomposed into two parts: the vertex and the edge calculation parts. 
		According to their computational complexity, we select four representative GNNs (GCN, GGNN, GAT, GaAN) for evaluation. 
		We breakdown their training time and memory usage, evaluate the effects of hyper-parameters, and assess the efficiency of the sampling techniques.
		The experimental evaluation indicates that the edge-related calculation is the performance bottleneck for most GNNs, dominating the training time and memory usage.
		Future optimization can focus on it. The sampling techniques are essential for training big graphs on GPUs, but their current implementations still have room for improvement.
	\end{abstract}

	\begin{keyword}
		graph neural network, performance bottleneck analysis, empirical evaluation, machine learning system, GPU
	\end{keyword}

\end{frontmatter}

\linenumbers

\section{Introduction}

In recent years, graph neural network is a hot reasearch topic in the field of artifical intelligence, 
and has achieved excellent performance on tasks such as node classification, link prediction and graph classification.\cite{comprehensive-survey-wu-2020}
\cite{zhou2018_gnn_review}, \cite{zhou2018_gnn_review}. These successes is related to graph strucure which reflects the vast majority of real-word data compared
to grid data strucure such as citation network and knowledge graph and deep learning's end-to-end learning capability.
At the same time, a series of parallel or distributed graph neural network systems have appeared.  
These systems abstract the graph neural network computing model from a large number of graph neural networks,
and design efficient implementations for the computing model with a lot of performance optimization.

\begin{itemize}
    \item message-passing general model \cite{gilmer_messgae_passing}. PyG and DGL is typical graph neural network system.
    The graph convolution operation is defined as the message operation function, reduce operation function and update operation function.
    PyG \cite{PyG} built upon PyTorch and obtained high data throught with sparse GPU accleration by providing dedicated CUDA kernels 
    and mini-batch technologies. DGL \cite{DGL} supported a variety of computing backends (Tensorflow, MXNet, PyTorch) and leveraged
    fusion kernel techniques, which combine message function with update function to provide further performance improvements compared to PyG \cite{PyG}. 
    \item SAGA-NN general model
    NeuGraph \cite{ma2019_neugraph} proposed SAGA-NN (Scatter-ApplyEdge-Gather-ApplyVertex with Neural Networks) programming model for graph neural network training.
    The SAGA-NN model divides the forward calculation of each layer in the graph neural network into four stages: Scatter, ApplyEdge, Gather, and ApplyVertex. 
    The ApplyEdge and ApplyVertex stages perform the calculation of the edge feature vector and vertex feature vector based on the neural network provided by the user. 
    Scatter and Gather are stages implicitly triggered by the NeuGraph system. These two stages prepare data for the ApplyEdge and ApplyVertex stages. 
    When programming, users only need to use the given operator to implement the ApplyEdge and ApplyVertex functions, and specify the Gather method,
    then they can use NeuGraph to automatically complete GNN training.
    \item Sample + Aggregate + Combine general model
    In the general GNN framework supported by AliGraph \cite{zhu2019_aligraph}, 
    each layer of GNN is disassembled into three basic operators: Sample, Aggregate and Combine. 
    Sample corresponds to sampling, Aggregate performs edge calculation, 
    and Combine corresponds to vertex calculation. 
    Because AliGraph is faced with actual large-scale graph data, AliGraph focuses on graph storage, 
    graph sampling, and graph calculation. In graph storage, the vertex-cut method is adopted, 
    that is, different edges are allocated to different machines. 
    In graph sampling, three sampling methods are supported, 
    Traverse: sampling a batch of vertices from a graph partition;
    Neighborhood: sample the 1-hop or multi-hop neighborhood of a vertex;
    Negative: generate negative sampling samples to accelerate convergence. 
    In particular, the weights in the Sampler also allow updating according to the gradient.
\end{itemize}

In the implementation of these graph neural network computing systems, 
different performance optimization techniques are used, 
but whether these performance optimization techniques really reflect the performance bottleneck
research in the GNN training process is still in doubt. 
At present, there is very little analysis of the specific performance bottlenecks of graph 
neural network training. Yan et al\cite{yan2020_analysis_gcns_gpu} analyzed the characteristics of GCN-like algorithms
in the inference phase with the classic graph analysis algorithm (PageRank) and MLP-based classic
The characteristics of the neural network were compared and analyzed, 
and it was found that the distribution of vertex degrees in the actual graph conforms
 to the idempotent distribution. Therefore, the vertex of the cache height may increase the hardware
cache hit rate, because vectorized atomic access can improve the efficiency of the aggregation stage, 
but this work only selects a specific GNN algorithm, which cannot well represent most of the GNN training
analysis. Zhange et al\cite{zhang2020_analysis_neugraph} experiment is based on the SAGA-NN programming model and DGL computing system,
the author believes that GNN has no fixed performance bottlenecks, performance bottlenecks will vary with different datasets and algorithms, 
but the degrees of the selected datasets is very low.Therefore, the performance bottleneck of GNN training deserves more research.

Since the most essential operation of each layer of the graph neural network can actually be summarized into two operations,
aggregate and update, the aggregate operation collects the information of neighboring vertices, 
and the edge operation is reflected on the graph; the update operation transforms the vertex information, 
which is reflected as vertex operation. Therefore, when studying the performance bottleneck of GNN training in this paper,
vertex operation and edge operation can be used as the basis for selecting typical algorithms to
study the performance bottleneck of GNN algorithm training. In survey\cite{comprehensive-survey-wu-2020}, \cite{zhou2018_gnn_review}, \cite{zhang2018_gnn_survey}, 
we performed point-edge complexity analysis on most GNN algorithms, and classified different algorithms into different quadrants. 
Four typical algorithms GCN, GGNN, GAT and GaAN are selected from the four quadrants. 
In the experiment, we selected six data sets with different characteristics, and considered the R-MAT generator to generate random graphs, 
and fixed the number of layers and common hyperparameters of the four algorithms. \\
To explore the bottleneck of GNN training, we transformed into discussing the following four issues:
Q1. How do hyperparameters affect training time and memory usage? Does it meet the time complexity analysis?\\
Q2. Which stage of GNN training is the most time-consuming?\\
Q3. Which factor has the greatest impact on memory in GNN training?\\
Q4. Can the sampling technique solve the performance bottleneck of GNN training?\\
For the above four problems, our Key Findings are:\\
For Q1, for each algorithm, we analyzed the impact of its various parameters on training time and memory usage. 
Experiments show that hyperparameters meet the time complexity analysis for training time and memory usage \\
For Q2, we can divide the training time into different levels: layer level, vertex/edge computation leval 
and basic operator leval, our training time of bareakdowon to different stages. We found that the characteristics of GNN
calculation are in line with GPU operations, and the bottleneck of GNN training is affected by the average degree.
In the real world, the average degree is more than 10 degrees, concentrated in the edge calculation part. 
When the calculation complexity is high, the performance bottleneck is concentrated on the basic operator of the calculation; 
while the calculation complexity is low, optimizing collect and aggreagte can improve performance \\
For Q3, we measured the impact of peak memory usage and expansion ratio on the scale, points and edges of input features dims, graphs. 
The peak memory usage of GNN during training can reach tens of times or even hundreds times, under fixed vertices, 
the peak memory of GNN increases linearly with the growth of edges; 
under the condition that the hyperparameters of the network structure are fixed, 
higher-dimensional input feature vectors can reduce the proportion of GPU memory expansion \\
For Q4, in order to verify the effectiveness of the sampling algorithm, we measured the change of training time
and memory with batch size. 
We found that the sampling technique can indeed greatly alleviate the memory problem, 
thereby extending the GNN algorithm to large-scale graphs. However, after the batching technique is increased, 
the sampling technique itself will cause more extra time, but it will be more time-consuming.\\
Through the above analysis, we provide the following suggestions for the GNN system:\\
1. The performance of hyperparameters in memory and training time is consistent with time complexity, 
we can theoretically analyze performance bottlenecks\\
2. GPU memory expansion is the most significant factor restricting the scalability of the GNN algorithm. 
When dealing with large-scale graphs, how to effectively save memory is always a priority; memory and edge complexity are linearly related \\
3. For real-world graphs, edge calculation is the time-consuming bottleneck. 
The collection, message, and aggregation of edge calculation are worth optimizing at each stage, especially the message stage. 
When the complexity of edge calculation is high, it is the most significant time-consuming; \\
4. The advantage of sampling technology is that it greatly reduces the overhead of peak memory usage, 
making large-scale graph neural network training possible, 
and the time-consuming optimization of sampling technology will also bring great gains to the entire training.
\section{Review of Graph Neural Networks}

In this section, we introduce the concepts related to the graph neural network and breifly survey typical graph neural networks.
We denote a simple graph $\mathcal{G}$ as $\mathcal{G}=(\mathcal{V}, \mathcal{E})$, where $\mathcal{V}$ and $\mathcal{E}$ are the vertex set and the edge set of $\mathcal{G}$, respectively.
Let $n=|\mathcal{V}|$ and $m=|\mathcal{E}|$ as the number of vertices/edges. 
We use $v_i$ $(0 \leq i < n)$ to denote a vertex and $e_{i,j}=(v_i, v_j)$ to denote the edge pointing from $v_i$ to $v_j$.
The adjacency set of $v_i$ is $\mathcal{N}(v_i)=\{v|(v_i, v) \in \mathcal{E}\}$.
We denote a \emph{vector} with a bold lower case letter like $\boldsymbol{x}$ and a \emph{matri}x with a bold upper case letter like $\boldsymbol{X}$.

\subsection{General Structure of Graph Neural Networks}

As illustracted in \figurename~\ref{fig:general_structure_of_gnn}, a typical GNN can be decomposed into three parts \cite{comprehensive-survey-wu-2020}: an input layer + several GNN layers + a prediction layer.

A GNN receives a graph $\mathcal{G}$ as the input.
Every vertex $v_i$ in $\mathcal{G}$ is attached with a feature vector $\boldsymbol{x}_i$ to describe the properties of the vertex.
The input layer of the GNN receives the feature vectors from all vertices and passes them to GNN layers.
A GNN layer consists of $n$ GNN units with each GNN unit corresponding to a vertex.
A GNN unit $v_i$ receives feature vectors from the GNN/input units of the previous layer that are adjacent to $v_i$ in $\mathcal{G}$.
For example, in \figurename~\ref{fig:general_structure_of_gnn}, 
A GNN outputs an embedding vector $\boldsymbol{o_i}$ for every vertex in $G$.
The GNN assumes that every vertex $v_i$ in $G$ is attached with a feature vector $\boldsymbol{x}_i$ in the input.
GNNs assume that every vertex in the feature vectors attached to every vertex of the 
The input layer receives feature vectors of different vertices.

\begin{figure}
	\centering
	\includegraphics[width=0.95\columnwidth]{figs/illustration/GNN_common_architecture.png}
	\caption{Structure of a typical graph neural network. (a) Demo GNN; (b) GNN unit; (c) Demo graph. The target application is the node classification. The demo GNN has two GNN layers.}
	\label{fig:general_structure_of_gnn}
\end{figure}

\section{Experiment Design}
\label{sec:experimental_design}

We design a series of experiments to find out the performance bottleneck in training graph neural networks.
We first introduce the experimental setting in Section~\ref{sec:experimental_env} and then give out our experimental scheme in Section~\ref{sec:experimental_scheme}.
The experiment results are presented and analyzed later in Section~\ref{sec:experiment_results}.

\subsection{Experimental Setting}
\label{sec:experimental_env}

\paragraph{Experimental Environment}
All the experiments were conducted in a CentOS 7 Linux server with kernel version 3.10.0.
The server had 40 cores and 90 GB main memory.
The server was equipped with an NVIDIA Tesla T4 GPU card with 16GB GDDR6 memory.
For the software environment, we adopted Python 3.7.7, PyTorch 1.5.0, and CUDA 10.1.
We implemented all the GNNs with PyTorch Geometric 1.5.0.

\paragraph{Dataset}
We used six real-world graph datasets as listed in \tablename~\ref{tab:dataset_overview} that were popular in the GNN accuracy evaluation \cite{yang2016_revisiting_semisupervised, zeng2020_graphsaint, shchur2018_pitfall_of_gnn}.
For directed graphs, PyG converts them into undirected ones during the data loading.
Thus, the average degree of a directed graph $\bar{d}=\frac{2|\mathcal{E}|}{|\mathcal{V}|}$.
For an undirected graph, $\mathcal{E}$ already contains two-direction edges and $\bar{d}=\frac{|\mathcal{E}|}{|\mathcal{V}|}$.
For the \texttt{cam} dataset, we generated random dense feature vectors.
Since we mainly focus on the training efficiency of GNNs instead of accuracies, we also used random graphs in the experiments.
To evaluate the effects of graph topological characteristics (like the average degree) on the performance bottleneck uniformly, we used the R-MAT graph generator \cite{rmat-generator}.
Input feature vectors of the random graphs were random dense vectors with a dimension of 32.
Vertices were divided into 10 classes randomly.

\begin{table}
    \centering
    \small
    \begin{tabular}{cccccccc}
        \toprule
        Dataset                                                 & $|\mathcal{V}|$ & $|\mathcal{E}|$ & $\bar{d}$ & $dim(\boldsymbol{x})$ & Sparsity & \#Class & Directed \\
        \midrule
        pubmed (pub) \cite{yang2016_revisiting_semisupervised}  & 19,717          & 44,324          & 4.5       & 500                   & 0.90     & 3       & Yes      \\
        amazon-photo (amp) \cite{shchur2018_pitfall_of_gnn}     & 7,650           & 119,081         & 31.1      & 745                   & 0.65     & 8       & Yes      \\
        amazon-computers (amc) \cite{shchur2018_pitfall_of_gnn} & 13,752          & 245,861         & 35.8      & 767                   & 0.65     & 10      & Yes      \\
        coauthor-physics (cph) \cite{shchur2018_pitfall_of_gnn} & 34,493          & 247,962         & 14.4      & 8415                  & 0.996    & 5       & Yes      \\
        flickr (fli) \cite{zeng2020_graphsaint}                 & 89,250          & 899,756         & 10.1      & 500                   & 0.54     & 7       & No       \\
        com-amazon (cam) \cite{yang2012_defining}               & 334,863         & 925,872         & 2.8       & 32                    & 0.0      & 10      & No       \\
        \bottomrule
    \end{tabular}
    \caption{Dataset overview. $\bar{d}$ represents the average vertex degree. $dim(\boldsymbol{x})$ is the dimension of the input feature vector. The sparsity is the proportion of zero elements in the input feature vectors.}
    \label{tab:dataset_overview}
\end{table}

\paragraph{Learning Task}
We used the node classification as the target task in GNNs due to its popularity in real-world applications.
We trained GNNs with the semi-supervised learning setting.
All vertices and their input feature vectors were used, but only parts of the vertices were attached with labels during the training and they were used to calculate the loss and gradients.
The vertices with unseen labels were used in the evaluation phase to check the accuracy of the current parameters.
%Since model parameters of GNNs were not restricted by the topological structure of $\mathcal{G}$, the model learned from the semi-supervised learning can directly extrapolate to unseen vertices.

\paragraph{GNN Implementation}
We implemented the four typical GNNs (GCN, GGNN, GAT, GaAN) with PyTorch Geometric 1.5.0.
To compare the performance characteristics of four GNNs side-by-side, we used a unified GNN structure for them: Input Layer $\rightarrow$ GNN Layer 0 $\rightarrow$ GNN Layer 1 $\rightarrow$ Softmax Layer (to prediction).
The structure was popular in the experimental evaluation of GCN \cite{kipf2017_gcn}, GAT \cite{huang2018_gat} and GaAN \cite{zhang2018_gaan}.
Since a GGNN layer requires the input and output hidden feature vectors have the same dimension, we added two MLP layers to transform the dimensions of the input/output feature vectors of the whole GNN: Input Layer $\rightarrow$ MLP $\rightarrow$ GGNN Layer 0 $\rightarrow$ GGNN Layer 1 $\rightarrow$ MLP $\rightarrow$ Softmax Layer.
We stored the dataset and the model parameters in the memory of GPU.
All the training was conducted on GPU.

\paragraph{Hyper-parameters}
We use $dim(\boldsymbol{v})$ to denote the dimension of a vector $\boldsymbol{v}$.
We picked the hyper-parameters of GNNs according to their original papers.
As GNNs used different hyper-parameters for different datasets, we picked the most popular hyper-parameters and used the same set of hyper-parameters for all the datasets in our experiments.
Some hyper-parameters like the dimensions of hidden feature vectors were common.
We set them to the same values in the four GNNs.
For GCN/GAT/GaAN, we set $\boldsymbol{h}^0_i = \boldsymbol{x}_i$, $dim(\boldsymbol{h}^1_i)=64$, and $dim(\boldsymbol{h}^2_i)=\#Classes$.
For GAT, we set the hyper-parameters according to \cite{huang2018_gat}.
The first GAT layer had 8 heads with $d^0_{head}=8$ and merged the heads by concatenating.
The second GAT layer used a single head with $d^1_{head}=d^1_{out}=\#Classes$.
For GGNN, since it uses extra MLP layers to transform the dimensions of the input/output feature vectors, we set $dim(\boldsymbol{h}^0_i) = dim(\boldsymbol{h}^1_i) = dim(\boldsymbol{h}^2_i) = 64$.
We used 8 heads in the both GaAN layers with $d_a=d_v=8$ and $d_m=64$.

\paragraph{Sampling Techniques}

We picked the neighbor sampler from GraphSAGE \cite{hamilton2017_graphsage} and the cluster sampler from ClusterGCN \cite{chiang2019_cluster_gcn} as the typical neighbor sampling and graph sampling techniques, respectively.
For the neighbor sampler, we used the neighborhood sample sizes (25 for GNN Layer 1, and 10 for GNN Layer 0) and the default batch size (512) from \cite{hamilton2017_graphsage}.
For the cluster sampler, we partitioned every input graph into 1500 partitions and used 20 partitions per batch according to \cite{chiang2019_cluster_gcn}.

\subsection{Experimental Scheme}
\label{sec:experimental_scheme}

To find out the performance bottleneck in GNN training, we conduct the bottleneck analysis with four questions.
The answers to those questions will give us a more comprehensive view of the performance characteristics of GNN training.
We design extensive experiments to find the answers empirically.
\begin{itemize}

    \item[Q1] \emph{How do the hyper-parameters affect the training time and the memory usage of a GNN?} (Section~\ref{sec:effects_of_hyper-parameters_on_performance})

          Every GNN has a group of hyper-parameters like the number of GNN layers and the dimension of hidden feature vectors. The hyper-parameters affect the training time per epoch and the peak memory usage during the training.
          To evaluate their effects, we measured how the training time per epoch and the peak memory usage (of GPU) changed as we increased the values of the hyper-parameters.
          Through the experiments, we want to verify the validity of the computational complexity analysis in \tablename~\ref{tab:gnn_overview_edge} and \tablename~\ref{tab:gnn_overview_vertex}.
          If the complexity analysis is valid, we can analyze the bottleneck theoretically.

    \item[Q2] \emph{Which stage is the most time-consuming stage in GNN training?} (Section~\ref{sec:training_time_breakdown})

          We can decompose the training time on different levels: layer level, vertex/edge computation level, and the basic operator level.
          On each level, we breakdown the training time of an epoch into several stages. The most time-consuming stage is the performance bottleneck.
          Optimizing its implementation can significantly reduce the training time.

    \item[Q3] \emph{Which consumes most of memory in GNN training?} (Section~\ref{sec:memory_usage_analysis})

          The limited memory capacity of a GPU card is the main factor preventing us from training GNNs on big graphs.
          We measured the peak memory usage in the GNN training under different graph scales, input feature dimensions, and average vertex degrees.
          Based on the results, we analyze which is the most memory-consuming component.
          Reducing its memory usage will enable us to train bigger GNNs under the same memory capacity.

    \item[Q4] \emph{Can the sampling techniques remove the performance bottleneck in GNN training?} (Section~\ref{sec:effects_of_sampling_techniques_on_performance})

          In theory, the sampling techniques can significantly reduce the graph neurons that participate in the training of a batch.
          Consequently, the training time and the memory usage should also decrease.
          To validate the effectiveness of the sampling techniques, we measured the training time and peak memory usage under different batch sizes.
          If the sampling techniques are effective, they are the keys to conduct GNN training on very big graphs.
          If they are not effective, we want to find out which impairs its efficiency.
\end{itemize}


\section{Evaluation Results and Analysis}
\label{sec:experiment_results}

We answer the four questions in Section~\ref{sec:experimental_scheme} one by one with experiments.
Without otherwise mentioned, the reported training time per epoch is the average wall-clock training time of 50 epochs, excluding abnormal epochs \footnote{During the training of some epochs, there are extra profiling overheads from NVIDIA Nsight Systems and GC pauses from the Python interpreter that significantly increase the training time. Assume Q1 and Q3 are the 25\% and 75\% quantiles of the training time of 50 epochs, respectively. We regard the epochs with the training time \emph{outside} the range of $[Q1 - 1.5 * (Q3-Q1), Q3 + 1.5 * (Q3-Q1)]$ as abnormal epochs.}.

\subsection{Effects of Hyper-parameters on Performance}
\label{sec:effects_of_hyper-parameters_on_performance}

According to \tablename~\ref{tab:gnn_overview_edge} and \tablename~\ref{tab:gnn_overview_vertex}, the time complexities of $\phi$ and $\gamma$ are linear to each hyper-parameter separately.
If we increase one of the hyper-parameters and fix the others, the training time should increase linearly.

To verify the time complexity analysis in \tablename~\ref{tab:gnn_overview_edge} and \tablename~\ref{tab:gnn_overview_vertex}, we first compare the training time of the four GNNs on the real-world datasets in \ref{tab:dataset_overview}.
The ranking of the training time is GaAN $\gg$ GAT $>$ GGNN $>$ GCN in all cases.
Since the real-world graphs have more edges than edges ($m > n$), the time complexity of the edge calculation stage affects more than the vertex calculation stage.
The ranking is consistent with the time complexity analysis.

\begin{figure}[htb]
    \centering
    \subfloat[\texttt{pub}\label{fig:exp_absolute_training_time_pubmed}]{\includegraphics[height=4cm]{figs/experiments/exp_absolute_training_time_comparison_pubmed.pdf}}
    \subfloat[\texttt{aph}\label{fig:exp_absolute_training_time_amazon-photo}]{\includegraphics[height=4cm]{figs/experiments/exp_absolute_training_time_comparison_amazon-photo.pdf}}
    \subfloat[\texttt{cph}\label{fig:exp_absolute_training_time_coauthor-physics}]{\includegraphics[height=4cm]{figs/experiments/exp_absolute_training_time_comparison_coauthor-physics.pdf}} \\
    \subfloat[\texttt{amc}\label{fig:exp_absolute_training_time_amazon-computers}]{\includegraphics[height=4cm]{figs/experiments/exp_absolute_training_time_comparison_amazon-computers.pdf}}
    \subfloat[\texttt{fli}\label{fig:exp_absolute_training_time_flickr}]{\includegraphics[height=4cm]{figs/experiments/exp_absolute_training_time_comparison_flickr.pdf}}
    \subfloat[\texttt{cam}\label{fig:exp_absolute_training_time_com-amazon}]{\includegraphics[height=4cm]{figs/experiments/exp_absolute_training_time_comparison_com-amazon.pdf}}
    \caption{Distribution of the wall-clock training time of 50 epoches on different datasets. GaAN crashed due to out of memory exception on the \texttt{cph} dataset.}
    \label{fig:exp_absolute_training_time}
\end{figure}

To further evaluate the effects of the hyper-parameters, we measure the training time of each GNN with varying hyper-parameters in \figurename~\ref{fig:exp_hyperparameter_on_vertex_edge_phase_time}.

\begin{figure}[tbp]
    \centering
    \subfloat[GCN\label{fig:exp_hyperparameter_on_vertex_edge_phase_time_gcn}]{\includegraphics[height=3cm]{figs/experiments/exp_hyperparameter_on_vertex_edge_phase_time_gcn.pdf}}
    %
    \subfloat[GGNN\label{fig:exp_hyperparameter_on_vertex_edge_phase_time_ggnn}]{\includegraphics[height=3cm]{figs/experiments/exp_hyperparameter_on_vertex_edge_phase_time_ggnn.pdf}}
    \\
    \subfloat[GAT\label{fig:exp_hyperparameter_on_vertex_edge_phase_time_gat}]{\includegraphics[height=6cm]{figs/experiments/exp_hyperparameter_on_vertex_edge_phase_time_gat.pdf}}
    \\
    \subfloat[GaAN\label{fig:exp_hyperparameter_on_vertex_edge_phase_time_gaan}]{\includegraphics[height=6cm]{figs/experiments/exp_hyperparameter_on_vertex_edge_phase_time_gaan.pdf}}

    \caption{Effects of hyper-parameters on the edge/vertex calculation time.}
    \label{fig:exp_hyperparameter_on_vertex_edge_phase_time}
\end{figure}


Since $dim(\boldsymbol{h^0})$ and $dim{\boldsymbol{h^1}}$ are determined by the dataset with $dim(\boldsymbol{h}^0=dim(\boldsymbol{x}))$ and $dim(\boldsymbol{h}^2)=\#Classes$,
for GCN and GGNN, the only modifiable hyper-parameter is the hidden dimension $dim(\boldsymbol{h}^1)$ with $dim(\boldsymbol{h}^1) = d^0_{out} = d^1_{in}$.
According to the time complexity analysis, if we fix other hyper-parameters but only increase $dim(\boldsymbol{h}^1)$, the computing costs of the GNN layer 0 and the GNN layer 1 both increase linearly with $dim(\boldsymbol{h}^1)$, causing the training time of the whole GNN also increasing linearly. 
\figurename~\ref{fig:exp_hyperparameter_on_vertex_edge_phase_time_gcn} and \figurename~\ref{fig:exp_hyperparameter_on_vertex_edge_phase_time_ggnn} show that the training time of GCN and GGNN increased linearly with $dim(\boldsymbol{h}^1)$ when $dim(\boldsymbol{h}^1)$ is big, consistent with the time complexity analysis.

For GAT, we modify the number of heads $K$ and the dimension of each head $d_{head}$ in the GAT layer 0.
The dimension of the hidden feature vector is determined correspondingly as $d^0_{out} = d^1_{in} = dim(\boldsymbol{h}^1) = K d_{head}$.
Thus, the computing costs of the GAT layer 0 and the GAT layer 1 increase linearly with K and $d_{head}$ separately. 
\figurename~\ref{fig:exp_hyperparameter_on_vertex_edge_phase_time_gat} confirms the theoretical analysis.

For GaAN, it is also based on the multi-head mechanism.
Its time complexity is affected by $dim(\boldsymbol{h}^1)$ ($d^0_{out} = d^1_{in} = dim(\boldsymbol{h}^1)$), $d_a$, $d_v$, $d_m$ and the number of heads $K$.
\figurename~\ref{fig:exp_hyperparameter_on_vertex_edge_phase_time_gaan} demonstrates that the training time increases linearly with the hyper-parameters, except for $dim(\boldsymbol{h}^1)$.
As $dim(\boldsymbol{h}^1)$ increases, the training time increases first slightly and then linearly.
We observe similar phenomena in GCN, GGNN, and GAT with low hyper-parameters.
When the hyper-parameters is too low, the GNN training cannot make full use of the computing power of the GPU.
When the hyper-parameter becomes high enough, the training time increases linearly, supporting the time complexity analysis.

\begin{figure}[tbp]
    \centering
    \subfloat[GCN]{\includegraphics[height=3cm]{figs/experiments/exp_hyperparameter_on_memory_usage_gcn.pdf}}
    \subfloat[GGNN]{\includegraphics[height=3cm]{figs/experiments/exp_hyperparameter_on_memory_usage_ggnn.pdf}}\\
    \subfloat[GAT]{\includegraphics[height=3cm]{figs/experiments/exp_hyperparameter_on_memory_usage_gat.pdf}}\\
    \subfloat[GaAN]{\includegraphics[height=3cm]{figs/experiments/exp_hyperparameter_on_memory_usage_gaan.pdf}}
    \caption{Effects of hyper-parameters on the peak GPU memory usage during the training, excluding the memory used by the dataset and the model parameters.}
    \label{fig:exp_hyperparameter_memory_usage}
\end{figure}

We further measured the effects of the hyper-parameters on the peak GPU memory usage in \figurename~\ref{fig:exp_hyperparameter_memory_usage}.
The memory usage also increases linearly as the hyper-parameters increase for all GNNs, except for GaAN on $dim(\boldsymbol{h}^1)$.
As the hidden feature vectors $\boldsymbol{h}^1$ consume a small proportion of memory in GaAN, the growth in the memory usage is not noticeable until $dim(\boldsymbol{h}^1)$ is large enough.

\paragraph{Summary}

The complexity analysis in \tablename~\ref{tab:gnn_overview_edge} and \tablename~\ref{tab:gnn_overview_vertex} is valid.
Fixing other hyper-parameters, each hyper-parameter itself affects the training time and the memory usage of a GNN Layer \textbf{in a linear way}.
Algorithm engineers can adjust hyper-parameters according to the time complexity to avoid explosive growth in the training time and memory usage.

\subsection{Training Time Breakdown}
\label{sec:training_time_breakdown}

To find out which stage/step dominates the training time, we decompose the training time and analyze the performance bottleneck level by level.

\subsubsection{Layer Level}

\begin{figure}[tbp]
    \centering
    \subfloat[GCN]{\includegraphics[height=4.6cm]{figs/experiments/exp_layer_time_proportion_gcn.pdf}}
    \subfloat[GGNN]{\includegraphics[height=4.6cm]{figs/experiments/exp_layer_time_proportion_ggnn.pdf}}\\
    \subfloat[GAT]{\includegraphics[height=4.6cm]{figs/experiments/exp_layer_time_proportion_gat.pdf}}
    \subfloat[GaAN]{\includegraphics[height=4.6cm]{figs/experiments/exp_layer_time_proportion_gaan.pdf}}
    \caption{Training time breakdown on the layer level. The training time of each layer includes the time spent on the forward, backward and evaluation phases. Each layer is further decomposed into the vertex and the edge calculation stages.}
    \label{fig:exp_vertex_edge_cal_proportion}
\end{figure}

\figurename~\ref{fig:exp_vertex_edge_cal_proportion} decomposes the training time of a GNN on the layer level.
The training time of each layer is the summation of the time in the forward, backward, and evaluation phases.
In GCN, GAT, and GaAN, the time spent on the layer 0 is much larger than the layer 1.
In those GNNs, the dimensions of the input/output feature vectors in the layer 0 are much larger than the dimensions in the layer 1.
$d^0_{in}=dim(\boldsymbol{x})$, $d^0_{out}=d^1_{in}=64$ and $d^1_{out}=\#Class$ and $dim(\boldsymbol{x}) \gg \#Class$.
For GaAN, since it requires the dimensions of the input/output feature vectors must be the same, the hyper-parameter are $d^0_{in}=d^0_{out}=d^1_{in}=d^1_{out}=64$ and the training time of both layers is close.

Each GNN layer can be further divided into the vertex and the edge calculation stages.
In \figurename~\ref{fig:exp_vertex_edge_cal_proportion}, GCN spends most of the training time on the edge calculation stage in most datasets.
A special case is \texttt{cph} dataset.
The dimension of the input feature vectors is very high in \texttt{cph}, making the vertex calculation stage of the GCN Layer 0 spend considerable time.
GGNN also spends the majority of its training time on the edge calculation stage.
But the high time complexity of its vertex updating function $\gamma$ makes the ratio of the vertex calculation in the total training time much higher than the other GNNs.
%In \textit{pub} and cam dataset, the edge calculation cost, and the vertex calculation cost are close for GGNN
%because the average degree of the two datasets is low (only 4.5 and 2.8).
For GAT and GaAN, due to their high edge calculation complexity, the edge calculation stage is the absolutely dominant stage.
In summary, \emph{the edge calculation stage is the most time-consuming stage in the GNN training}.

\begin{figure}[tbp]
    \centering
    \subfloat[GCN]{\includegraphics[height=4cm]{figs/experiments/exp_avg_degree_on_vertex_edge_cal_time_gcn.pdf}}
    \subfloat[GGNN]{\includegraphics[height=4cm]{figs/experiments/exp_avg_degree_on_vertex_edge_cal_time_ggnn.pdf}}\\
    \subfloat[GAT]{\includegraphics[height=4cm]{figs/experiments/exp_avg_degree_on_vertex_edge_cal_time_gat.pdf}}
    \subfloat[GaAN]{\includegraphics[height=4cm]{figs/experiments/exp_avg_degree_on_vertex_edge_cal_time_gaan.pdf}}
    \caption{Effects of the average degree on the time proportion of the edge/vertex calculation. Graphs were generated with the R-MAT generator by fixing the number of vertices as 50,000. }
    \label{fig:exp_avg_degree_on_vertex_edge_cal_time}
\end{figure}

The experimental results also indicate that the average degree of the dataset affects the time-consuming proportion of the edge/vertex calculation time.
For GaAN, the time spent on the vertex calculation stage exceeds the edge calculation stage on the \texttt{pub} and \texttt{cam} datasets, because the average degrees of the two datasets are low, making $|\mathcal{E}|$ and $|\mathcal{V}|$ much closer.
To evaluate the effects of the average degree, we used the R-MAT generator to generate random graphs with 50k vertices and the average degrees ranging from 2 to 100.
\figurename~\ref{fig:exp_avg_degree_on_vertex_edge_cal_time} shows the training time of the four GNNs under different average degrees.
As the average degree increases, the training time of the edge calculation stage grows \emph{linearly}.
For GCN, GAT, and GaAN, the edge calculation stage dominates the entire training time even under small average degrees.
Only for GGNN that has high vertex and low edge calculation complexities, the training time of the vertex calculation stage exceeds the edge calculation stage under low average degrees ($<5$).
Therefore, \emph{improving the efficiency of the edge calculation stage is the key to reduce the GNN training time}.

\subsubsection{Step Level in Edge Calculation}

\begin{figure}[tbp]
    \centering
    \includegraphics[width=1\columnwidth]{figs/illustration/steps_in_edge_calculation.pdf}
    \caption{Step decomposition of the edge calculation in the GNN layer $l$.}
    \label{fig:steps_in_edge_calculation}
\end{figure}

In the implementation of PyG, the edge calculation stage can be decomposed into four steps: collection, messaging, aggregation, and updating, as shown in \figurename~\ref{fig:steps_in_edge_calculation}.
The edge index is a matrix with $M$ rows and 2 columns that holds the edge set of the graph, where $M=|\mathcal{E}|$.
The two columns of the matrix store the source vertex and the target vertex of each edge, respectively.
The collection step copies the vertex feature vectors from the previous layer $\boldsymbol{h}_i^l$ to the ends of each edge
in the edge index, to form the parameters tensor $[\boldsymbol{h}^l_i, \boldsymbol{h}^l_{j}, \boldsymbol{e}^l_{i,j}]$ of the messaging function $\phi$.
This step only involves the data movement.
The messaging step calls the messaging function $\phi$ to get message vectors of all edges $\boldsymbol{m}_{i, j}^l$.
The aggregation step aggregates the message vectors with the same target vertex into an aggregated vector $\boldsymbol{s}^l_i$ with the aggregation operation $\Sigma$.
The updating step is optional.
It performs an additional transformation on the aggregated vectors (for example, adding bias in GCN).
The aggregated vectors $\boldsymbol{s}^l_i$ (after the updating step) will be fed into the vertex updating function $\gamma$ as one of the input parameters.

\begin{figure}[tbp]
    \centering
    \subfloat[GCN]{\includegraphics[height=4cm]{figs/experiments/exp_edge_calc_decomposition_gcn.pdf}}
    \subfloat[GGNN]{\includegraphics[height=4cm]{figs/experiments/exp_edge_calc_decomposition_ggnn.pdf}}\\
    \subfloat[GAT]{\includegraphics[height=4cm]{figs/experiments/exp_edge_calc_decomposition_gat.pdf}}
    \subfloat[GaAN]{\includegraphics[height=4cm]{figs/experiments/exp_edge_calc_decomposition_gaan.pdf}}
    \caption{Training time breakdown of the edge calculation stage (including both GNN layers).}
    \label{fig:exp_edge_calc_decomposition}
\end{figure}

We decompose the execution time of the edge calculation stage in \figurename~\ref{fig:exp_edge_calc_decomposition}.
In each GNN, the proportions of the four steps are rather stable, rarely affected by datasets. 
For GAT and GaAN with the high edge calculation complexity, the messaging step consumes most of the training time. 
For GCN and GGNN with the low edge complexity, the proportions of the steps are close. 
Since the messaging function $\phi$ of GGNN uses the pre-computed $\hat{\boldsymbol{h}}^l_i$ as the message vector directly, the time spent on the messaging step of GGNN is negligible.
Although the collecting step does not conduct any computation and only involves data movement, it occupies noticeable execution time in all the GNNs.
The experiments show that \emph{the performance bottleneck on the step level depends on the complexity of the messaging function $\phi$}.
For GNNs with the high complexity, the messaging function $\phi$ is the performance bottleneck.
Optimizing the implementation of $\phi$ can significantly reduce the training time.
For the other GNNs, optimization should focus on reducing the costs of the collection and the aggregation steps.
Additionally, improving the efficiency of the collection step can benefit all GNNs.

\subsubsection{Operator Level}

The functions $\phi$, $\Sigma$ and $\gamma$ in the vertex and edge calculation are made up of a series of basic operators implemented on the GPU, like the matrix multiplication \texttt{mm}, the elementwise multiplication \texttt{mul} and the index-based selection \texttt{index\_select}.
\figurename~\ref{fig:exp_top_basic_ops} shows the five most time-consuming basic operators in each GNN, averaged over all the real-world graphs in \tablename~\ref{tab:dataset_overview}.

\begin{figure}[tbp]
    \centering
    \subfloat[GCN]{\includegraphics[height=4cm]{figs/experiments/exp_top_basic_ops_gcn.pdf}}
    \subfloat[GGNN]{\includegraphics[height=4cm]{figs/experiments/exp_top_basic_ops_ggnn.pdf}}\\
    \subfloat[GAT]{\includegraphics[height=4cm]{figs/experiments/exp_top_basic_ops_gat.pdf}}
    \subfloat[GaAN]{\includegraphics[height=4cm]{figs/experiments/exp_top_basic_ops_gaan.pdf}}
    \caption{Top 5 time-consuming basic operators of typical GNNs. The time proportion of each basic operator is averaged over all graphs with the error bar indicating the maximal and the minimal.}
    \label{fig:exp_top_basic_ops}
\end{figure}

\paragraph{GCN}
The most time-consuming basic operator is the matrix multiplication \texttt{mm} used in the vertex updating function $\gamma$.
The elementwise multiplication \texttt{mul} used in the messaging function $\phi$ is also time-consuming.
The other three operators are used in the edge calculation stage: \texttt{scatter\_add\_} for the aggregation step in the forward phase, \texttt{gather} for the aggregation step in the backward phase, and \texttt{index\_select} for the collection step.
For GCN, the basic operators related to the edge calculation stage consume the majority of the training time.

\paragraph{GGNN}
The top basic operator is \texttt{mm} used in the vertex updating function $\gamma$.
Due to its high time complexity, the proportion of the \texttt{mm} is much higher than the other operators.
The \texttt{thnn\_fused\_gru\_cell} operator used in the backward phase of $\gamma$ is also noticeable.
The other three operators are used in the edge calculation stage.

\paragraph{GAT}
All the top basic operators except for \texttt{mm} are related to the edge calculation stage.
The \texttt{mm} operator is used in the vertex updating function $\gamma$.

\paragraph{GaAN}
The top basic operator is \texttt{bmm} used in the messaging function $\phi$.
The \texttt{addmm} operator and the \texttt{mm} operator are used in both the vertex and the edge calculation stages, where the edge calculation stage is dominant.

In general, the most time-consuming operator in the four GNNs, is still the matrix multiplication \texttt{mm} and the elementwise multiplication \texttt{mul}, \emph{making GNN training suitable for GPUs}.
Although the aggregation step in the edge calculation stage is relatively simple (like sum and mean), the related operators \texttt{scatter\_add} and \texttt{gather} still consume a certain amount of the time.
The operators have to synchronize between hardware threads to avoid updating the same aggregated vector at the same time. They also conduct non-regular memory access with the access pattern determined by the edge set dynamically.
For GPUs, they are less efficient than \texttt{mm}.
The index-based selection \texttt{index\_select} operator from the collection step consume around 10\% of the time in all GNNs.
Though GPUs have high on-chip memory bandwidth, improving the efficiency of \texttt{scatter\_add}/\texttt{gather}/\texttt{index\_select} can benefit the training of all kinds of GNNs.

\paragraph{Summary}
The GNN training is suitable for GPUs.
\textbf{The edge calculation stage is the main performance bottleneck in most cases}, except for training GNNs with high vertex calculation complexity on low-average-degree graphs.
The performance bottleneck in the edge calculation stage depends on the time complexity of the messaging function $\phi$.
\begin{itemize}
    \item If the time complexity of $\phi$ is \textbf{high}, the \textbf{efficiency of $\phi$} limits the performance. Reducing its computation cost (via optimizing its implementation or modifying the algorithm) can significantly reduce the training time.
    \item If the time complexity of $\phi$ is \textbf{low}, the \textbf{collection step} and the \textbf{aggregation step} limit the performance. The collection step involves lots of data movement. The aggregation step suffers from data synchronization and non-regular data access. Optimizing their implementations can significantly reduce the training time.
\end{itemize}

\subsection{Memory Usage Analysis}
\label{sec:memory_usage_analysis}

During the GNN training, all data (including datasets and intermediate results) are stored in the on-chip memory of the GPU.
Compared with the main memory on the host side, the capacity of the GPU memory is very limited.
\emph{The GPU memory capacity limits the scales of the graphs that a GPU can train GNNs on}.
For example, GaAN is unable to train on \texttt{cph} dataset due to the out of memory exception.

\begin{figure}[tbp]
    \centering
    \includegraphics[height=5cm]{figs/experiments/exp_memory_usage_stage_amp.pdf}
    \caption{Memory usage of each phase. Dataset: \texttt{amp}.}
    \label{fig:exp_memory_usage_stage_amp}
\end{figure}

\begin{figure}[tbp]
    \centering
    \includegraphics[height=4.5cm]{figs/illustration/ggnn_vertex_func_computation_graph.pdf}
    \caption{Computation graph of the vertex updating function $\gamma$ of GGNN.}
    \label{fig:ggnn_vertex_func_computation_graph}
\end{figure}

\figurename~\ref{fig:exp_memory_usage_stage_amp} shows the peak memory usage of each phase in the GNN training on the \texttt{amp} dataset.
The trend is similar on the other datasets.
\emph{The GNN training achieves its peak memory usage in the forward and the backward phases}.
The forward phase generates lots of intermediate results.
Some key intermediate results are cached, increasing memory usage.
The cached results are used in the gradient calculation in the backward phase.
\figurename~\ref{fig:ggnn_vertex_func_computation_graph} shows the computation graph of the vertex updating function $\gamma$ of GGNN.
The computation graph has a large number of operators. Each operator generates an intermediate tensor.
Some key intermediate tensors are cached.
The cached tensors are the main source of memory usage in the loss phase.
By the end of the backward phase, the cached tensors are released.
Since the evaluation phase does not need to calculate the gradients, it does not cache intermediate tensors.
Its memory usage declines sharply.

\begin{figure}[tbp]
    \centering
    \includegraphics[height=5cm]{figs/experiments/exp_memory_expansion_ratio.pdf}
    \caption{Memory expansion ratios of typical GNNs.}
    \label{fig:exp_memory_expansion_ratio}
\end{figure}

The peak memory usage during the GNN training far exceeds the size of the dataset itself.
We define the \emph{memory expansion ratio} (MER) as the ratio of the peak memory usage during the training to the memory usage after loading the dataset.
\figurename~\ref{fig:exp_memory_expansion_ratio} compares MER of different GNNs.
GCN has the lowest MER (up to 15) while GaAN has the highest MER (up to 104).
\emph{The high MER limits the data scalability of GNNs}, making GPUs unable to handle big graphs.

\begin{figure}[tbp]
    \centering
    \includegraphics[height=5cm]{figs/experiments/exp_memory_expansion_ratio_input_feature_dimension_com-amazon.pdf}
    \caption{Memory expansion ratio under different dimensions of the input feature vectors. Dataset: \texttt{cam}.}
    \label{fig:exp_memory_expension_ratio_input_feature_dimension}
\end{figure}

\figurename~\ref{fig:exp_memory_expansion_ratio} also indicates that the same GNN has different MERs for different datasets.
Two characteristics of a dataset affect the MER: the dimension of the input feature vector and the average degree.

Given the same graph, the scales of the intermediate results are mainly affected by the hyper-parameters of the GNN.
If the dimension of the input feature vectors is high (like the \texttt{cph} dataset), the size of the dataset is large.
The size may become comparable to the scales of the intermediate results,  making the MER low.
To find out how the dimension affects the MER, we generated random input feature vectors with different dimensions for the \texttt{cam} dataset and measured the MER in \figurename~\ref{fig:exp_memory_expension_ratio_input_feature_dimension}.
For a GNN under the same hyper-parameters, \emph{the MER decreases as the dimension of the input feature vectors increases}.

\begin{figure}[tbp]
    \centering
    \subfloat[Peak memory usage]{\includegraphics[height=4cm]{figs/experiments/exp_memory_expansion_ratio_input_graph_number_of_edges_peak_memory.pdf}}
    \subfloat[Memory expansion ratio]{\includegraphics[height=4cm]{figs/experiments/exp_memory_expansion_ratio_input_graph_number_of_edges_expansion_ratio.pdf}}
    \caption{Memory usage under different average degrees of the graph. The graph was generated with the R-MAT generator fixing the number of vertices at 10K and the dimension of the input feature vectors at 32.}
    \label{fig:exp_memory_expansion_ratio_input_graph_number_of_edges}
\end{figure}

The average degree of the graph also affects MER by influencing the relative scales of the intermediate results from the edge/vertex calculation stages.
Fixing the number of vertices $|\mathcal{V}|$, we used the R-MAT generator to generate random graphs with different average degrees.
\figurename~\ref{fig:exp_memory_expansion_ratio_input_graph_number_of_edges} shows how the memory usage changes according to the average degrees.
As the average degree $\bar{d}$ increases, the number of edges $|\mathcal{E}|$ increases and the peak memory usage increases \emph{linearly} with $\bar{d}$.
The edge calculation stage gradually dominates the memory usage and \emph{the MER converges to a stable value}.
The stable value is determined by the complexity of the edge calculation stage.
Except for GGNN, the MER of the other GNNs increases as $\bar{d}$ increases.
As GGNN has high vertex calculation complexity, the MER related to the vertex calculation stage is much larger than the edge calculation stage.
When the edge calculation stage dominates the memory usage, its MER becomes smaller.

\begin{figure}[tbp]
    \centering
    \subfloat[Peak memory usage]{\includegraphics[height=4cm]{figs/experiments/exp_memory_expansion_ratio_input_graph_number_of_vertices_fixed_edge_peak_memory.pdf}}
    \subfloat[Memory expansion ratio]{\includegraphics[height=4cm]{figs/experiments/exp_memory_expansion_ratio_input_graph_number_of_vertices_fixed_edge_expansion_ratio.pdf}}
    \caption{Memory usage under different numbers of vertices of the graph. The graph was generated with the R-MAT generator fixing the number of edges at 500K and the dimension of the input feature vectors at 32.}
    \label{fig:exp_memory_expansion_ratio_input_graph_number_of_vertices_fixed_edge}
\end{figure}

We also fixed the number of edges $|\mathcal{E}|$ in the graphs and generated random graphs with different $|\mathcal{V}|$.
\figurename~\ref{fig:exp_memory_expansion_ratio_input_graph_number_of_vertices_fixed_edge} shows how the memory usage changes according to $|\mathcal{V}|$.
All GNNs are insensitive to the changes in $|\mathcal{V}|$ compared to $|\mathcal{E}|$.
Except for GGNN, the MERs of the other GNNs decline as $|\mathcal{V}|$ increases because the scales of the datasets increase more quickly than the scales of the intermediate results.
As GGNN has high vertex calculation complexity, the scales of the intermediate results are much more sensitive to $|\mathcal{V}|$.
It indicates that \emph{the intermediate results of the edge calculation stage dominated the memory usage during the GNN training}.

\paragraph{Summary}
The \emph{high} memory expansion ratio severely restrictes the data scalability of the GNN training.
The memory usage mainly comes from the intermediate results of the \emph{edge calculation stage}.
Fixing the number of vertices, the memory usage increases \emph{linearly} along with the number of edges.
Optimizing the memory usage of the edge calculation stage can significantly reduce the memory expansion ratio.
Fixing the GNN structures and the hyper-parameters, increasing the dimension of the input feature vectors can also reduce the memory expansion ratio.

\subsection{Effects of Sampling Techniques on Performance}
\label{sec:effects_of_sampling_techniques_on_performance}

With the sampling techniques, GNNs can be trained in a mini-batch manner.
Each mini-batch updates the model parameters based on a small subgraph sampled from the original input graph.
Thus, the training time per batch and the peak memory usage during the training should both decline significantly.

In the implementation in PyG, the GNN model and the dataset resides on the GPU side.
To process each epoch, PyG samples the original dataset in the main memory and generates several batches.
Each batch is a small subgraph of the dataset.
To train on each batch, PyG sends the sampled subgraph to the GPU, calculates the gradients on the subgraph, and updates the model parameters directly on the GPU.
With the sampling techniques, the model parameters are updated by a stochastic gradient descent optimizer.
PyG conducts the evaluation phase every several epochs (either on the CPU side or the GPU side) to determine whether to stop the training.
In this section, the experiments focus on the training phase of each batch.

\begin{figure}[tbp]
    \centering
    \subfloat[Neighbor sampler]{\includegraphics[height=4cm]{figs/experiments/exp_sampling_minibatch_realtive_graph_info_graphsage_gcn.pdf}} \\
    \subfloat[Cluster sampler]{\includegraphics[height=4cm]{figs/experiments/exp_sampling_minibatch_realtive_graph_info_cluster_gcn.pdf}}
    \caption{Sizes of the sampled subgraphs under different batch sizes. Each batch size was sampled 50 times and the average value was reported. The error bar indicates the standard deviation. The batch size is relative to the full graph.}
    \label{fig:exp_sampling_minibatch_graph_info}
\end{figure}

\figurename~\ref{fig:exp_sampling_minibatch_graph_info} shows how the size of the sampled subgraph changes with the batch size.
For the neighbor sampler, the relative batch size is the proportion of the sampled vertices of the last GNN layer in $|\mathcal{V}|$.
For the cluster sampler, the relative batch size is the proportion of the sampled partitions to all partitions of the graph.
The neighbor sampler is very sensitive to the batch size.
As the batch size increases, the size of the sampled subgraph first increases quickly and then stabilizes.
The cluster sampler is much less sensitive compared to the neighbor sampler.
The number of vertices and the average degree of the sampled subgraphs increases linearly with the batch size.

\begin{figure}[tbp]
    \centering
    \includegraphics[width=0.4\columnwidth]{figs/experiments/exp_sampling_minibatch_degrees_distribution_amazon-photo.pdf}
    \caption{Vertex degree distribution of the sampled subgraph (relative batch size: 6\%) and the original graph. Dataset:\texttt{amp}.}
    \label{fig:exp_sampling_minibatch_degrees_distribution}
\end{figure}

It is worth noting that the average degree of the sampled subgraph is \emph{much lower} than the average degree of the whole graph, especially when the relative batch size is low.
Taking the neighbor sampler with the relative batch size of 6\% as an example, the average degree of the \texttt{amp} dataset is 31.1, but the average degree of the sampled subgraph is only 5.8.
For the cluster sampler, the average degree is 3.0.
\figurename~\ref{fig:exp_sampling_minibatch_degrees_distribution} compares the degree distribution of the sampled subgraphs with the original graph.
The slopes of the curves are similar.
It indicates that the sampled subgraphs still follow the power-law degree distribution.
However, there are much less vertices in the sampled subgraphs, significantly lowering the average degrees.
According to the experimental results in Section~\ref{sec:training_time_breakdown}, if the average degree becomes lower, the proportion of the training time spent on the vertex calculation stage will become higher, especially for GGNN.

\begin{figure}[tbp]
    \centering
    \subfloat[Neighbor sampler on \texttt{amc}]{\includegraphics[height=5cm]{figs/experiments/exp_sampling_relative_batch_size_train_time_stack_graphsage_amazon-computers.pdf}}
    \subfloat[Neighbor sampler on \texttt{fli}]{\includegraphics[height=5cm]{figs/experiments/exp_sampling_relative_batch_size_train_time_stack_graphsage_flickr.pdf}} \\
    \subfloat[Cluster sampler on \texttt{amc}]{\includegraphics[height=5cm]{figs/experiments/exp_sampling_relative_batch_size_train_time_stack_cluster_amazon-computers.pdf}}
    \subfloat[Cluster sampler on \texttt{fli}]{\includegraphics[height=5cm]{figs/experiments/exp_sampling_relative_batch_size_train_time_stack_cluster_flickr.pdf}}
    \caption{Training time per batch breakdown. FULL means that the full graph participates in the training.}
    \label{fig:exp_sampling_batch_train_time}
\end{figure}

To find out the performance bottleneck with the sampling techniques, we decompose the training time per batch into three phases: \emph{sampling} on the CPU, \emph{transferring} sampled subgraphs from the CPU to the GPU and \emph{training} with the subgraphs on the GPU.
\figurename~\ref{fig:exp_sampling_batch_train_time} shows the time breakdown of the four GNNs under different relative batch sizes.
For the neighbor sampler, the sampling technique reduces the training time per batch only when the batch size is very small.
When the batch becomes bigger, the sampling and the data transferring phases introduce noticeable overheads, making the training time exceed the full-batch training.
For the clustering sampler, the sampled subgraph is smaller than the neighbor sampler under the same relative batch size.
The reduction in the training time is more obvious than the neighbor sampler.
However, the overheads increase quickly as the relative batch size increase.
The training time under the 25\% relative batch size already exceede the time of full-batch training.
The experimental results indicate that the current implementation of the sampling techniques in PyG is inefficient.
When the batch size is slightly big, more than 50\% of the time has been spent on sampling and data transferring.
\emph{The sampling techniques are only efficient under small batch sizes.}

\begin{figure}[tbp]
    \centering
    \subfloat[Neighbor sampler on \texttt{amc}]{\includegraphics[height=4cm]{figs/experiments/exp_sampling_memory_usage_relative_batch_size_graphsage_amazon-computers_peak_memory.pdf}}
    \subfloat[Neighbor sampler on \texttt{fli}]{\includegraphics[height=4cm]{figs/experiments/exp_sampling_memory_usage_relative_batch_size_graphsage_flickr_peak_memory.pdf}} \\
    \subfloat[Cluster sampler on \texttt{amc}]{\includegraphics[height=4cm]{figs/experiments/exp_sampling_memory_usage_relative_batch_size_cluster_amazon-computers_peak_memory.pdf}}
    \subfloat[Cluster sampler on \texttt{fli}]{\includegraphics[height=4cm]{figs/experiments/exp_sampling_memory_usage_relative_batch_size_cluster_flickr_peak_memory.pdf}}
    \caption{Peak memory usage under different batch sizes. FULL means the full graph participates in the training.}
    \label{fig:exp_sampling_memory_usage}
\end{figure}

The main advantage of the sampling technique is \emph{reducing the peak memory usage} during the training.
\figurename~\ref{fig:exp_sampling_memory_usage} shows the memory usage under different batch sizes.
The peak memory usage declines significantly even under big batch sizes.
The sampling techniques make training GNNs on big graphs possible for GPUs.

\begin{figure}[tbp]
    \centering
    \includegraphics[height=5cm]{figs/experiments/exp_small_graph_train_time.pdf}
    \caption{Training time per epoch on small random graphs. For each number of vertices, we generated 50 random R-MAT graphs with the average degree of 4.0 and reported the average training time per epoch (without the evaluation phase). The error bar indicates the standard deviation.}
    \label{fig:exp_small_graph_train_time}
\end{figure}

The disadvantage of the sampling technique is wasting GPU resources.
As the sampling techniques are only effective under small batch sizes, the sampled subgraphs will be very small in those cases.
They cannot make full use of the computing power of a GPU.
To simulate the situation, we generated random graphs with few vertices and measured the training time per epoch in \figurename~\ref{fig:exp_small_graph_train_time}.
As the number of vertices increases, the training time is almost unchanged except for GaAN.
The training time of GaAN increases only with $|\mathcal{V}| \geq 4000$.


\paragraph{Summary}
The sampled subgraphs have lower average degrees than the whole graph.
With small batch sizes, the sampling techniques can significantly reduce the training time per batch and the peak memory usage during the training.
However, small batch sizes cannot make full use of the computing power of a GPU.
With big batch sizes, the current implementation of the sampling techniques in PyG is inefficient.
The time spent on the sampling phase and the data transferring phase even exceeds the training phase.

\section{Insights}
\label{sec:insights}

Through the extensive experiments, we propose the following key findings/suggestions for how to optimize the performance of GNN training.

\begin{enumerate}
    \item \emph{The time complexity in \tablename~\ref{tab:gnn_overview_edge} and \tablename~\ref{tab:gnn_overview_vertex} points out the performance bottleneck theoretically.}
          The experimental results validate the time complexity analysis.
          The time complexity points out where the bottleneck comes from.
          Optimization should focus on complex operations in $\phi$ and $\gamma$.

    \item \emph{The computational cost of a GNN layer is mainly affected by the dimensions of the input and the output hidden feature vectors.}
          Theoretically and empirically, the training time and the memory usage both increase \emph{linearly} with the dimensions of the hidden feature vectors.
          GNNs are friendly to high-dimensional scenarios.
          Algorithm engineers can use high-dimensional feature vectors to improve the expressive power of a GNN without worrying explosive growth in the training time and memory usage.

    \item \emph{Performance optimizations should focus on improving the efficiency of the edge calculation.}
          The edge calculation is the most time-consuming phase in most GNNs.
          \begin{itemize}
              \item If the complexity of the message function $\phi$ is high, the implementation of $\phi$ is critical to performance.
                    Improving its efficiency can significantly reduce the training time.
                    For example, the attention mechanism in GNNs (like GAT and GaAN) requires an extra sub-layer to calculate the attention weight of each edge.
                    Implementing it with the specially optimized basic operators on GPU is a potential optimization.
              \item If the complexity of $\phi$ is low, the efficiency of the collect step and the aggregation step becomes critical.
                    The existing GNN libraries \cite{DGL, PyG, ma2019_neugraph} already introduce the \emph{fused} operator to improve their efficiency.
                    When the message function $\phi$ is an assignment or a scalar multiplication of the hidden feature vector of the source vertex, the libraries replace the collect, message and aggregate steps with a single fused operator.
                    The fused operator calculates the aggregated vectors directly from the input hidden feature vectors, minimizing the memory footprints and overlapping the memory accessing with calculation.
                    By this way, it significantly reduces the training time of GNNs with low edge calculation complexity (like GCN) \cite{yan2020_characterizing_gcn, zhang2020_analysis_neugraph}.
                    However, the applicable condition of the fused operator is very restrict.
                    It does not work for $\phi$ with more complex operations like matrix multiplication.
                    A potial optimization is proposing an implementation of the edge calculation that generates the aggregated vectors directly from the input hidden feature vectors on the fly, without materializing the parameter vectors and the message vectors.
          \end{itemize}
    \item \emph{The high memory usage caused by the intermediate results of the edge calculation limits the data scalability of the GNN training.}
          The memory expansion ratios of the typical GNNs are very high, making GPU unable to handle big graphs.
          To train GNNs with big datasets, one solution is to distribute the dataset among severl GPUs and frequently swap parts of the dataset between GPUs and the main memory \cite{ma2019_neugraph}.
          Another possible solution \cite{chen2016_training_deep} comes from the DNN training.
          It only checkpoints key intermediate results during the forward propagation and recalculates the missing results on demand during the backproporgation.
          Implementing the checkpoint mechanism in the GNN training is another potential optimization.

    \item \emph{Sampling techniques can significantly reduce the training time and the memory usage, but its implementation is still inefficient}.
          The sampling techniques are effective under small batch sizes.
          Its current implementation brings considerable overheads when the batch size becomes large.
          Improving the efficiency of the sampling is a potentil optimization.
          The subgraphs sampled with small batch sizes are small.
          They cannot make fully use of the computing power of a GPU.
          How to improve the GPU utilization under small batch sizes is another problem to solve.
          One possible solution is to train multiple batches asynchronously on the same GPU and use the asynchronous stochastic gradient descent to speed up the converge.

\end{enumerate}

\section{Related Work}

In this section, we brefily review the graph neural network models survey (GNNs survey), 
graph neural network system(GNN system) and GNNs performance bottleneck analysis work.

\subsection{GNNs survey}
GNNs survey: Zhou et al \cite{zhou2018_gnn_review}, Zhang et al \cite{zhou2018_gnn_review} and Wu et al \cite{comprehensive-survey-wu-2020} 
surveyed existing graph neural works models in different categorizations and viewpoints and provided a comprehensive review about GNNs' model 
architecture, principle and application. Dwived et al \cite{dwivedi2020_benchmark_of_gnn} discussed the performance of different GNN algorithms 
under the graph dataset of different scenes, and gived the key operators of designing effective GNNs in different scenarios. 
However, these efforts are focused on the GNN algorithm itself and accuracy, without a detailed analysis of memory usage and running time.

\subsection{GNN system}
PyG \cite{PyG}, DGL \cite{DGL} designed the graph neural network system based on the message-passing model. 
PyG \cite{PyG} built upon PyTorch and obtained high data throught with sparse GPU accleration by providing dedicated CUDA kernels 
and mini-batch technologies. DGL \cite{DGL} supported a variety of computing backends (Tensorflow, MXNet, PyTorch) and leveraged
fusion kernel techniques, which combine message function with update function to provide further performance improvements compared to PyG \cite{PyG}. 
NeuGraph \cite{ma2019_neugraph} proposed a new programming model, SAGA-NN(Scatter-ApplyEdge-Gather-ApplyVertex with Neural Networks), and achieved excellent
performance in a single machine with mutliple GPUs by using graph computation optimization such as data partitioning management, scheduling and parall mutli-gpu processing.
AliGraph \cite{zhu2019_aligraph} targetd at Attributed Heterogeneous Graph data model, disassembed each layer of GNN into there basic operations: Sampling, Aggregate and Combine,
and leveraged graph parition, spearate storage of attributes and the importance-base vertex caching strategy in storage layer. 
In general, various optimization techniques are used in these GNN systems' design, but whether these optimization techniques match the bottlenecks of GNN performance is a question
worth studying.

\subsection{GNNs performance bottleneck analysis work}
Yan et al \cite{yan2020_analysis_gcns_gpu} compared the performance characteristics of GNN algorithms, Graph Processing and MLP-based Neural Network, provided guidelines in software optimizaton and hardware optimization.
Zhang et \cite{zhang2020_analysis_neugraph} characterized the GNN computation at the inference stage based on SAGA-NN (Scatter-ApplyEdge-Gather-ApplyVertex with Neural Networks) programming model, 
and found that GNN computation has no fixed performance bottlenecks, and believed that each part has the value of optimization. 
Compared to their works, the classification angle of our work is unique, and neither of their works consider the graph scalabilty and the access of sampling technology,
which coverred in this paper.
\section{Conclusion and Future Work}
\label{sec:conclusion}

In this work, we systematically explore the performance bottlenecks in graph neural network training and inference.
%
We model the existing GNNs with the message-passing framework. 
%
We classify the GNNs according to their edge and vertex calculation complexity to select four typical GNNs for evaluation. 
%
The experimental results validate our complexity analysis.
%
Fixing other hyper-parameters, the training time, inference time, and memory usage increase linearly with each hyper-parameter of the four GNNs.
%
To find out the performance bottlenecks in the training/inference time, we decompose the training/inference time per epoch on different levels.
%
The time breakdown analysis indicates that the edge calculation stage and its related basic operators are the performance bottlenecks for most GNNs.
%
Moreover, the intermediate results produced by the edge calculation stage cause high memory usage, limiting the data scalability.
%
Adopting sampling techniques can reduce the memory usage of training and inference significantly, without sacrificing accuracy. 
%
However, the current implementation of the sampling techniques in PyG brings considerable sampling overheads.
%
The small sampled subgraphs cannot make full use of the computing power of a GPU card either.
% 
Our analysis indicates that the edge calculation stage should be the main target of optimizations.
%
Reducing its memory usage and improving its efficiency can significantly improve the performance of GNN training and inference.
%
Based on the analysis, we propose several potential optimizations for the GNN libraries/systems.
%
We believe that our analysis can help developers to have a better understanding of the characteristics of GNN training and inference.

In this work, we analyze performance bottlenecks in a \emph{single-GPU} environment on \emph{static} graphs with the \emph{message-passing} framework.
%
Performance bottlenecks in multi-GPU/distributed GNN training/inference, dynamic graphs and other GNN frameworks are also worth studying.
%
In the future, we plan to extend our analysis of performance bottlenecks in GNN training/inference with the following directions:
%
\begin{enumerate}
    \item \emph{Multi-GPU or distributed environment.}
    %
    To handle large-scale graph datasets, multi-GPU and distributed training/inference are necessary.
    %
    These methods will inevitably introduce overheads such as inter-GPU and inter-machine communication. 
    %
    How these overheads affect performance bottlenecks is worthy to focus on.
    %
    \item \emph{Spatial-temporal graph datasets.}
    %
    Spatial-temporal graphs have dynamic topology structures.
    %
    They appear in a variety of applications like traffic speed forecasting \cite{li2018_DCRNN} and human action recognition \cite{yan2018_STGCN}.
    %
    %Learning hidden patterns from spatial-temporal graphs become increasingly important.
    %
    Many new GNNs are proposed to handle this kind of graph.
    %
    How the performance bottlenecks of these GNNs are different from the classic GNNs is also worthy of in-depth study.
    %
    %Whether spatial-temporal graphs will affect performance is worthy of our attention.
    %
    \item \emph{Other GNN frameworks.}
    %
    In this work, we analyzed with the popular message-passing framework.
    %
    Some emerging GNN learning systems also adopt different GNN frameworks like SAGA framework \cite{ma2019_neugraph} and edge-centric framework \cite{he2019_EnGN}.
    %
    Whether different frameworks lead to different performance bottlenecks is worth discussing.
\end{enumerate}
\section*{Acknowledgements}

This work is funded in part by the National Key R\&D Program of China (2019YFC1711000), the China NSF Grants (No. U1811461, 62072230), the Collaborative Innovation Center of Novel Software Technology and Industrialization, Alibaba Innovative Resarch Project, and the program B for outstanding PhD candidate of Nanjing University.



\nocite{*}% Show all bib entries - both cited and uncited; comment this line to view only cited bib entries;
\bibliography{gnnref.bib}

\end{document}
