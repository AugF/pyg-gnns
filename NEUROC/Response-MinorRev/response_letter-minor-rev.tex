% !TEX program=pdflatex
% LaTeX rebuttal letter example. 
% 
% Copyright 2019 Friedemann Zenke, fzenke.net
%
% Based on examples by Dirk Eddelbuettel, Fran and others from 
% https://tex.stackexchange.com/questions/2317/latex-style-or-macro-for-detailed-response-to-referee-report
% 
% Licensed under cc by-sa 3.0 with attribution required.

\documentclass[12pt]{article}
\usepackage[utf8]{inputenc}
\usepackage{lipsum} % to generate some filler text
\usepackage{fullpage}
\usepackage{times}
\usepackage{amsmath, amsthm, amsfonts}
\usepackage{booktabs}
\usepackage{makecell}
\usepackage{multirow}
\usepackage{lscape}
\usepackage{xcolor}
\usepackage{float}
\usepackage{subfig}
\usepackage{graphicx}
\usepackage{framed}
\usepackage{soul}
\usepackage[colorlinks=true, linkcolor=blue, citecolor=black]{hyperref}
\usepackage{natbib}
% Vector
\newcommand{\MyVec}[1]{\boldsymbol{#1}}
% Matrix
\newcommand{\MyMat}[1]{\boldsymbol{#1}}
% Vectors with hat
\newcommand{\hvec}[1]{\hat{\boldsymbol{h}}_{#1}}
% Model parameters
\newcommand{\Param}[1]{\textcolor{blue}{#1}}

\usepackage{blindtext}
\usepackage[most]{tcolorbox} 
\definecolor{block-gray}{gray}{0.95}


\newtcolorbox{zitat}[2][]{%
    colback=block-gray,
    grow to right by=-0mm,
    grow to left by=-0mm, 
    boxrule=0pt,
    boxsep=0pt,
    breakable,
    enhanced jigsaw,
    borderline west={4pt}{0pt}{gray},
    title={#2\par},
    colbacktitle={block-gray},
    coltitle={black},
    fonttitle={\bfseries},
    attach title to upper={},
    #1,
}

\newenvironment{myquote}[1]%
{\vspace{0.5em}\begin{zitat}{#1}}
{\end{zitat}\vspace{0.5em}}


% import Eq and Section references from the main manuscript where needed
% \usepackage{xr}
% \externaldocument{manuscript}

% package needed for optional arguments
\usepackage{xifthen}
% define counters for reviewers and their points
\newcounter{reviewer}
\setcounter{reviewer}{0}
\newcounter{point}[reviewer]
\setcounter{point}{0}

% This refines the format of how the reviewer/point reference will appear.
\renewcommand{\thepoint}{P\,\thereviewer.\arabic{point}} 

% command declarations for reviewer points and our responses
\newcommand{\reviewersection}{\stepcounter{reviewer} \bigskip \hrule \setcounter{figure}{0}\section*{Reviewer \thereviewer}}

\newenvironment{point}
   {\refstepcounter{point} \bigskip \noindent {\textbf{Reviewer~Point~\thepoint} } ---\ \begin{sf}}
   {\end{sf} \par}

\newcommand{\shortpoint}[1]{\refstepcounter{point}  \bigskip \noindent 
	{\textbf{Reviewer~Point~\thepoint} } ---\begin{sf} ~#1 \end{sf}\par}

\newenvironment{reply}
   {\medskip \noindent \textbf{Reply}:\  }
   {\medskip}

\newcommand{\shortreply}[2][]{\medskip \noindent \textbf{Reply}:\  #2
	\ifthenelse{\equal{#1}{}}{}{ \hfill \footnotesize (#1)}%
	\medskip}

% Numbering figures according to reviewers
\renewcommand{\thefigure}{\thereviewer{}-\arabic{figure}}

\begin{document}

\section*{Response to the Reviewers}
% General intro text goes here
We thank the reviewers for their insightful comments on our work.
%
We have made modifications to our manuscript according to Reviewer 1's comments. 

% Let's start point-by-point with Reviewer 1
\reviewersection

% R1Q0
\begin{point}
    In the revised manuscript, the authors have addressed all my concerns, especially adding more computation efficiency discussion. Current version looks very impressive. Just a small issue left. Some discussions about limitations or future efforts should be better added.
\end{point}

\begin{reply}
    Thank you for the insightful suggesion.
    %
    According to the suggestion, we have added a paragraph at the end of Section 7 ``Conclusion and Future Work'' to discuss limitations of our work and points out potential future research directions.
    %
    In this work, we mainly analyze performance bottlenecks of GNN training/inference in a \emph{single-GPU} environment on \emph{static} graphs with the \emph{message-passing} framework.
    %
    Performance bottlenecks in \emph{multi-GPU}/\emph{distributed} GNN training/inference with \emph{dynamic} graphs and \emph{other} GNN frameworks are also worth studying.
    %
    In the future, we plan to extend our work in the following directions:
    %
\begin{enumerate}
    \item \emph{Multi-GPU and distributed GNN training/inference.}
    %
    To handle large-scale graph datasets, training/inferring GNNs with multiple GPUs or in a distributed environment is necessary.
    %
    Multi-GPU and distributed GNN training/inference will inevitably introduce overheads such as inter-GPU and inter-machine communication. 
    %
    How these overheads affect performance bottlenecks is worthy to focus on.
    %
    \item \emph{Spatial-temporal graph datasets.}
    %
    Spatial-temporal graphs have dynamic topology structures.
    %
    They appear in a variety of applications like traffic speed forecasting [\cite{li2018_DCRNN}] and human action recognition [\cite{yan2018_STGCN}].
    %
    %Learning hidden patterns from spatial-temporal graphs become increasingly important.
    %
    Many new GNNs are proposed to handle this kind of dynamic graphs.
    %
    How the performance bottlenecks of these GNNs are different from the classic GNNs is also worthy of in-depth study.
    %
    %Whether spatial-temporal graphs will affect performance is worthy of our attention.
    %
    \item \emph{Other GNN frameworks.}
    %
    In this work, we conducted analysis with the message-passing framework that is popular among existing GNN learning systems.
    %
    Some emerging GNN learning systems also adopt different frameworks like SAGA framework [\cite{ma2019_neugraph}] and edge-centric framework [\cite{he2019_EnGN}].
    %
    Whether different frameworks lead to different performance bottlenecks is worth further investigation.
\end{enumerate}

\end{reply}

\bibliographystyle{apalike}
\bibliography{../gnnref.bib}

\end{document}
