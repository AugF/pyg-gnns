\section{Conclusion}
\label{sec:conclusion}

In this work, we systematically study the performance bottleneck in graph neural network training.
We model the existing GNNs with the message-passing framework.
We classify the GNNs according to their edge and vertex calculation complexities to pick four typical GNNs to evaluate.
The experimental results validate our complexity analysis.
The training time and the memory usage increase linearly with the hyper-parameters of a GNN.
To find out the performance bottleneck in time, we decompose the training time per epoch on different levels.
The training time breakdown indicates that the edge calculation and its related basic operators are the performance bottleneck for most GNNs.
Moreover, the intermediate results produced during the edge calculation cause high memory usage, limiting the data scalability.
Adopting sampling techniques can reduce training time and memory usage significantly.
However, the current implementation brings considerable sampling overheads.
The small sampled graphs cannot make full use of the computing power of a GPU either.
Our analysis indicates that the edge calculation should be the main target of optimizations.
Reducing its memory usage and improving its efficiency can significantly improve the performance of GNN training.
Based on the analysis, we propose several potential optimizations for the GNN learning frameworks.
We believe that our analysis can help developers to have a better understanding of the characteristics of GNN computation.
