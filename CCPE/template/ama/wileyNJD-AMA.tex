
\documentclass[AMA,STIX1COL]{WileyNJD-v2}
\usepackage{lineno,hyperref}
\usepackage{amsmath, amsthm}
\usepackage{relsize}
\usepackage{booktabs}
\usepackage{makecell}
\usepackage{lscape}
\usepackage{xcolor}
\usepackage{subfig}
\usepackage{float}

\renewcommand{\thefootnote}{\fnsymbol{footnote}}

\articletype{Article Type}%

\received{}
\revised{}
\accepted{}
        
\raggedbottom

\begin{document}

\title{This is the sample article title}

\author[1]{Author One*}
\author[2,3]{Author Two}
\author[3]{Author Three}

\authormark{AUTHOR ONE \textsc{et al}}


\address[1]{\orgdiv{Org Division}, \orgname{Org Name}, \orgaddress{\state{State name}, \country{Country name}}}
\address[2]{\orgdiv{Org Division}, \orgname{Org Name}, \orgaddress{\state{State name}, \country{Country name}}}
\address[3]{\orgdiv{Org Division}, \orgname{Org Name}, \orgaddress{\state{State name}, \country{Country name}}}

\corres{*Corresponding author name, This is sample corresponding address. \email{authorone@gmail.com}}

\presentaddress{This is sample for present address text this is sample for present address text}

\abstract[Summary]{The graph neural network (GNN) has become a popular research area for its state-of-the-art performance in many graph analysis tasks.
Recently, various graph neural network libraries have emerged.
They make the development of GNNs convenient, but their performance on large datasets is not satisfying.
In this work, we analyze the performance bottleneck in training GNN with GPUs empirically.
A GNN layer can be decomposed into two parts: the vertex and the edge calculation parts.
According to their computational complexity, we select four representative GNNs (GCN, GGNN, GAT, GaAN) for evaluation.
We breakdown their training time and memory usage, evaluate the effects of hyper-parameters, and assess the efficiency of the sampling techniques.  
The experimental evaluation indicates that the edge-related calculation is the performance bottleneck for most GNNs, dominating the training time and memory usage.
Future optimization can focus on it. 
The sampling techniques are essential for training big graphs on GPU, but their current implementations still have room for improvement.}

\keywords{graph neural network, performance bottleneck analysis, empirical evaluation, machine learning system, GPU}

\maketitle


\section{Macros}

This URL contains the LaTeX Class file for Software: Practice and Experience (SP\&E):
\begin{verbatim}
\url{https://onlinelibrary.wiley.com/page/journal/1097024x/homepage/la_tex_class_file.htm}
\end{verbatim}
Clicking on ``AMA-stix.zip'' downloads the file AMA-stix-1510051231000.zip from this URL:
\begin{verbatim}
\url{https://wol-prod-cdn.literatumonline.com/pb-assets/assets/1097024X/AMA-stix-1510051231000.zip}
\end{verbatim}
In file AMA-stix/ama/wileyNJD-AMA.tex, there is an example latex document for authors.
I have edited this file to demonstrate problems with the latex macros in files WileyNJD-v2.cls and WileyNJD-AMA.bst, and used the following command to generate a PDF file.
\begin{verbatim}
pdflatex wileyNJD-AMA; bibtex wileyNJD-AMA; pdflatex wileyNJD-AMA; pdflatex wileyNJD-AMA;
\end{verbatim}


\section{First Problem}

The header on this page is:
\begin{quote}
Received {\color{red}26 April 2016} | Revised {\color{red}6 June 2016} | Accepted {\color{red}6 June 2016}
\end{quote}
Authors do not know these dates, so they are likely to just leave the dates unchanged.
It would be better to have a placeholder:
\begin{quote}
Received {\color{red}Added at production} | Revised {\color{red}Added at production} | Accepted {\color{red}Added at production}
\end{quote}


\section{Second Problem}

SP\&E style use symbols for footnotes and numbers for bibliography citations.
However, the latex macros generate numbers for footnotes rather than symbols number, which makes it impossible to tell the difference between a footnote and a citation.
This is a footnote\footnote{Footnote} and this is a citation~\cite{book01}, and they both are numbered $^1$.
Attempting to fix the problem by adding the following latex command results in an error.
\begin{verbatim}
\renewcommand{\thefootnote}{\fnsymbol{footnote}}
\end{verbatim}


\section{Third Problem}\label{section}

The ``ref'' macro used to print label for section, figures and tables leaves an extra space after the label number for figures and tables but not sections.
Figure~\ref{figure} has an extra space after the number, as does Table~\ref{table}, but Section~\ref{section} does not have an extra space.

\begin{figure}
\centering
I'm a figure above the caption
\caption{This is the sample figure caption}
\label{figure}
\end{figure}

\begin{table}
\caption{This is the sample table caption}
\label{table}
\centering
I'm a table below the caption
\end{table}


\section{Fourth Problem}

There are multiple problems in the WileyNJD-AMA.bst macros for formatting bibliographic entries.
Specifically, fields within entries are missing, as is space between fields, possibly in an attempt to shorten the length of the bibliographic entries.

Here is an exhaustive list of all fields for all bibliography entries.
\begin{quote}
article~\cite{article01}, book~\cite{book01}, booklet~\cite{booklet01}, inbook~\cite{inbook01}, incollection~\cite{incollection01}, inproceedings~\cite{inproceedings01}, manual~\cite{manual01}, mastersthesis~\cite{mastersthesis01}, misc~\cite{misc01}, phdthesis~\cite{phdthesis01}, proceedings~\cite{proceedings01}, techreport~\cite{techreport01}, unpublished~\cite{unpublished01}
\end{quote}
Looking at the references below, there are multiple fields missing from the bibliographic entries, there are spaces before periods, and there are fields run together, etc.

\nocite{*}% Show all bib entries - both cited and uncited; comment this line to view only cited bib entries;
\bibliography{wileyNJD-AMA}%

\end{document}
